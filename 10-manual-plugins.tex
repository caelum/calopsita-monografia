\section{Apêndice II - Criando plugins}

A criação de plugins no calopsita é baseada em pontos de extensão. Os pontos de extensão já implementados são:

\begin{itemize}
	\item{\textbf{Gadget} - Implemente uma entidade do Hibernate com essa interface, e então você pode adicionar informações
		aos cartões que possuirem esse gadget. Por exemplo se você quiser adicionar uma estimativa de velocidade, tempo total 
		ou dificuldade do cartão}
	\item{\textbf{Transformer} - Implemente essa interface e anote a classe gerada com @Component. Assim você consegue
		modificar as listagens, por exemplo mudando a ordenação ou removendo determinados itens.}
	\item{\textbf{PluginConfig} - Implemente essa interface e anote a classe gerada com @Component. Com ela é possível
		adicionar itens aos menus do sistema, baseados nos parâmetros da requisição}
\end{itemize}

Além disso, você pode criar novas telas para o sistema. Para isso basta criar um controlador do 
VRaptor~\footnote{http://vraptor.caelum.com.br/documentacao} e as respectivas jsps de resultado.
Todas as classes geradas precisam estar abaixo do pacote \textbf{br.com.caelum.calopsita} para que o
VRaptor consiga enxergá-las.

Por exemplo, se fôssemos fazer um plugin para estimar velocidade dos cartões, precisamos criar as classes:

\begin{itemize}
	\item{o Gadget para adicionar informação ao cartão:
		\begin{lstlisting}
			package br.com.caelum.calopsita.plugins.velocidade;
			
			@Entity
			public class VelocidadeCard implements Gadget {
				@Id
				@GeneratedValue
				private Long id; // o id do banco
				
				@OneToOne
				private Card card; // o cartão que esse gadget vai adicionar informação
				
				private Integer velocidade; // a informação a mais
				
				// getters e setters
			}
		\end{lstlisting}	
	}
	\item{um Transformer para ordenar os cartões por velocidade:
		\begin{lstlisting}
			package br.com.caelum.calopsita.plugins.velocidade;
			@Component
			public class OrdenaPorVelocidadeTransformer implements Transformer<Card> {

				public boolean accepts(Class<?> type) {
					return type.equals(Card.class); // vai transformar listas de cartões
				}

				public List<Card> transform(List<Card> list, Session session) {
					Collections.sort(list, new VelocidadeComparator()); // ordena a lista de acordo com o comparator abaixo
					return list;
				}

				public static class VelocidadeComparator implements Comparator<Card> {
					public int compare(Card esquerda, Card direita) {
						// pega o gadget do tipo dado do cartão. Null se o cartão não tiver o gadget
						VelocidadeCard velocidadeEsquerda = esquerda.getGadget(VelocidadeCard.class); 
						VelocidadeCard velocidadeDireita = direita.getGadget(VelocidadeCard.class);
						
						// decide se o cartão da esquerda é maior que o da direita
						// de acordo com o contrato do comparator
						if (velocidadeEsquerda == null) {
							return 1;
						} else if (velocidadeDireita == null) {
							return -1;
						}
						return velocidadeEsquerda.getVelocidade() - velocidadeDireita.getVelocidade();
					}
				}
			}

		\end{lstlisting}
	}
	
	\item{um Controlador do VRaptor que vai tratar as requisições para esse plugin:
	
		\begin{lstlisting}
			package br.com.caelum.calopsita.plugins.velocidade;
			
			@Resource
			public class VelocidadeController {
				
				private Result result;
				public VelocidadeController(Result result) {
					this.result = result;
				}
				// seguindo o padrão das urls
				@Path("/projects/{project.id}/velocidade")
				@Get
				public List<Card> estima(Project project) {
					return project.getAllCards();
				}
	
				@Path("/projects/{velocidadeCard.card.project.id}/velocidade")
				@Post
				public void adiciona(VelocidadeCard velocidadeCard) {
					// salva o velocidadeCard no banco
					// redireciona para a estimativa de cartões
					result.use(logic()).redirectTo(VelocidadeController.class).estima(velocidadeCard.getCard().getProject());
				}
				
			}
		\end{lstlisting}
	}
	
	\item{um jsp que responda mostre a tela de estimar cartões. Logo precisa estar em /WEB-INF/jsp/velocidade/estima.jsp
		\begin{verbatim}
			<!-- estilo da página, organização e etc -->
			<c:forEach items="${cardList}" var="card">
				<form action="<c:url value="/projects/${project.id}/velocidade" method="POST">
					<!-- campos do formulário passar valores para a lógica -->
				</form>"
			</c:forEach>
		\end{verbatim}
	}
\end{itemize}

Para instalar o plugin, basta colocar as classes criadas no classpath (dentro de um jar, ou no WEB-INF/classes),
jsps na pasta /WEB-INF/jsp, e eventuais javascripts e css's na pasta web.


