\section{Motivação e público alvo}

A motivação maior de se construir um sistema para gerenciamento de projetos ágeis e seu público alvo estão intimamente relacionados. 

Em aplicações comerciais, uma das maiores reclamações com relação à adoção de métodos ágeis é de que é impossível ter um cliente presente a todo tempo, por mais acessível que ele seja. Se houvesse uma forma de o cliente se manter informado com o andamento do seu \software{} e priorizar os próximos cartões \textit{online}, essa barreira seria eliminada.

A eventual ausência do cliente não é o único problema que o desenvolvimento ágil enfrenta hoje. Equipes distribuídas estão cada vez mais comuns -- desenvolvedores que trabalham em pares em cima de um código e conversam através da telecolaboração. Este assunto tem sido objeto de estudo de Frederick P. Brooks e será abordado em seu próximo livro~\cite{brooks}. Lidar com equipes distribuídas requer uma centralização das informações do projeto acessível de qualquer lugar do mundo.

Um outro grande exemplo da necessidade de trabalhar num mesmo projeto com pessoas de qualquer parte do mundo é o desenvolvimento \opensource{}. Sistemas de \textit{tickets}\footnote{http://www.atlassian.com/software/jira/} são muito usados nesse nicho, mas muitos deles deixam a desejar ou são complexos demais para se entender.

Para tantos públicos, a limitação a uma determinada metodologia, seus métodos e métricas não é viável. É um desejo poder personalizar o \calopsita{} de acordo com as necessidades dos usuários e, assim, desde o começo, houve uma grande atenção em não pensar apenas numa metodologia. As ferramentas atualmente vistas no mercado oferecem soluções para metodologias específicas, mas deixam a desejar na adaptabilidade.

Embora o \calopsita{} seja mais uma ferramenta, sua intenção é ser um facilitador das interações entre os indivíduos de um determinado projeto, indo de acordo com um dos valores do Manifesto Ágil~\cite{manifesto}:

\begin{quote}
\textit{``Individuals and interactions over processes and tool''}
\end{quote}

Nessa linha, partiu-se de uma ideia de desenvolver um sistema adaptável e flexível, que se molde às necessidades de cada equipe, independente dos métodos adotados, e que seja capaz de unir uma equipe distribuída.

O objetivo é atender a todas essas necessidades, ou ser facilmente extensível, mantendo, no processo, um código limpo e do qual se tem orgulho. Sendo um trabalho acadêmico, a equipe arriscou ousar em alguns pontos e ir além do que é visto no mercado atual. Esses pontos serão mostrados e explicados no decorrer desta monografia.
