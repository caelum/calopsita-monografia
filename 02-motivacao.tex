\section{Motivação e público alvo}

A motivação maior de se contruir um sistema para gerenciamento de projetos ágeis e seu público alvo estão intimamente relacionados. 

Em aplicações comerciais, uma das maiores reclamações com relação à adoção de métodos ágeis é de que é impossível ter um cliente presente a todo tempo, por mais acessível que ele esteja. Se houvesse uma forma de o cliente se manter informado com o andamento do seu \software e priorizar os próximos cartões online, ajudaria.

E esse não é o único problema que o desenvolvimento ágil enfrenta hoje. Equipes distribuidas estão cada vez mais comuns -- desenvolvedores que pareiam em código e conversam através da telecolaboração de que tem falado Frederich Brooks, e será um dos assuntos de seu próximo livro~\cite{brooks}. Lidar com equipes distribuidas requer uma centralização das informações do projeto acessível de qualquer lugar do mundo.

Um outro grande exemplo da necessidade de tabalhar num mesmo projeto com pessoas de qualquer parte do mundo é o desenvolvimento \opensource. Sistemas de tickets são muito usados nesse nicho, mas muitos deles deixam a desejar ou são complexos de mais para se entender.

Para tantos públicos, a limitação a uma determinada metodologia, seus métodos e métricas não é viável. Gostaríamos de poder personalizar o Calopsita de acordo com as necessidades dos que utilizam e, assim, desde o começo, houve uma grande atenção em não focar numa metodologia apenas -- que é o que o mercado já oferece hoje.

Embora o Calopsita seja mais uma ferramenta, sua intenção é ser um facilitador das interações entre os indivíduos de um determinado projeto, indo de acordo com um dos valores do Manifesto Ágil~\cite{manifesto}:

``Individuals and interactions over processes and tool''

Nessa linha, partimos de uma idéia de desenvolver um sistema adaptável e flexível, que se adapte às necessidades de cada equipe, independente dos métodos adotados e que seja capaz de unir uma equipe distribuida.

E o objetivo é atender a todas essas necessidades, ou ser facilmente extensível, mantendo, no processo, um código limpo e do qual nos orgulhemos. Sendo um trabalho acadêmico, arriscamos ousar em alguns pontos e ir além do que vemos no mercado atual. Falaremos desses pontos no decorrer desta monografia.
