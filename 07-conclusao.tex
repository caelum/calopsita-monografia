\section{Conclusão}

O \calopsita{} tem por missão ajudar times ágeis que trabalhem de maneira distribuída. Isso, é claro, sem obrigar
a adaptação de todo um time a uma ferramenta nova, que atrapalha no processo já estabelecido e que funciona. Times
diferentes tem diferentes necessidades e uma ferramenta que force o fluxo de trabalho a ser diferente do que o time 
está acostumado pode ter um impacto negativo no desenvolvimento de um projeto. O \calopsita{} ataca esse problema
sendo o mais flexível possível, se ajustando a diferentes necessidades.

\subsection{O futuro}

Mesmo com a apresentação deste trabalho, pretendemos continuar trabalhando no \calopsita{}. Ainda existem muitas funcionalidades
e plugins que queremos implementar, entre eles:

\begin{itemize}
	\item{Ordem de cartões: quando mais de um cartão de uma mesma prioridade, precisamos de uma ordem entre os cartões de uma
	mesma iteração para determinar qual deve ser escolhido primeiro. Isso impacta inclusive na geração do Kanban}
	\item{Gráfico de burnup}
	\item{Kanban}
	\item{integração com sistemas de versionamento, como github}
	\item{criação de feed RSS para recebimento de atualizações e movimentações de cartões}
\end{itemize}

Além disso, é provável que o projeto seja um dos escolhidos para fazer parte da disciplina de Programação Extrema a ser
ministrada no primeiro semestre de 2010.