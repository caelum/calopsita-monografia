\section{Conclusão}

O \calopsita{} nasceu com o propósito de ajudar times ágeis distribuidos. Seu diferencial em relação às demais ferramentas está no fato de não ser restrito a uma única metodologia, de ser adaptável ao fluxo de trabalho de qualquer time (através de sub-cartões), de ter uma interface mais agradável e de ser bastante extensível (através de sua arquitetura de plugins). De acordo com Hugo Corbucci, nosso cliente, o Calopsita já está muito adiantado em relação ao \textit{Xplanner}, ferramenta de mesmo projeto amplamente utilizada:

\begin{quote}
``A usabilidade é inquestionável, o \calopsita{} é um projeto que mostra CLARAMENTE uns 15 anos de avanço em cima do \textit{Xplanner} do ponto de vista da usabilidade.''
\end{quote}

******* aqui dividem duas tentativas de conclusão. as duas estão ruims. rola um merge?? *********

O \calopsita{} tem por missão ajudar times ágeis que trabalhem de maneira distribuída. Isso, é claro, sem obrigar a adaptação de todo um time a uma ferramenta nova, que atrapalha no processo já estabelecido e que funciona. Times diferentes tem diferentes necessidades e uma ferramenta que force o fluxo de trabalho a ser diferente do que o time está acostumado pode ter um impacto negativo no desenvolvimento de um projeto. O \calopsita{} ataca esse problema sendo o mais flexível possível, se ajustando a diferentes necessidades.

Também é claro para nós que o \calopsita{} não substitui o tradicional Kanban. Por isso, o foco é equipes distribuídas.


\subsection{O futuro}

Mesmo com a apresentação deste trabalho, pretendemos continuar trabalhando no \calopsita{}. Ainda existem muitas funcionalidades e plugins que queremos implementar, entre eles:

\begin{itemize}
	\item{Ordem de cartões: quando mais de um cartão de uma mesma prioridade, precisamos de uma ordem entre os cartões de uma mesma iteração para determinar qual deve ser escolhido primeiro. Isso impacta inclusive na geração do Kanban}
	\item{Gráfico de burnup}
	\item{Kanban}
	\item{integração com sistemas de versionamento, como github}
	\item{criação de feed RSS para recebimento de atualizações e movimentações de cartões}
	\item{marcação de horas}
	\item{templates de metodologias}
	\item{personas}
\end{itemize}

Além disso, é provável que o projeto seja um dos escolhidos para fazer parte da disciplina de Programação Extrema a ser ministrada no primeiro semestre de 2010.
