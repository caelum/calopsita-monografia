\section{Conclusão}

Embora já esteja sendo utilizado há 6 meses em projetos pessoais e comerciais, não apenas pela equipe, mas também por outras equipes e desenvolvedores, o \calopsita{} ainda tem muito para onde crescer -- e é intenção que o projeto cresça.

O plano para o próximo ano é que mais empresas adotem o \calopsita{} para o gerenciamento de suas aplicações. Para esse objetivo, há um trabalho de divulgação da solução para diversos grupos de desenvolvimento de grandes empresas que já utilizam métodos ágeis.

Além disso, o \calopsita{} será um dos projetos desenvolvidos na matéria de programação extrema no próximo semestre do IME. Com isso, além de continuar o desenvolvimento do projeto, pretende-se colocar os alunos em contato com diferentes métodos e \textit{frameworks} adotados no mundo.

Entre os ítens de maior valor para os clientes, os seguintes \textit{plugins} serão implementados muito em breve:

\begin{itemize}
	\item{Ordem de cartões: cartões de uma mesma prioridade devem poder ter precedência sobre outros de uma mesma iteração;}
	\item{Gráfico de burnup: para marcar o andamento de uma iteração, um gráfico de acompanhamento das tarefas prontas no decorrer do tempo.}
	\item{Kanban: o quadro branco usado por diversas metodologias para manter a equipe informada do que acontece no projeto.}
\end{itemize}

A respeito do que foi desenvolvido durante todo o ano e do suporte que a equipe recebeu das muitas pessoas que se envolveram no projeto, muitos agradecimentos devem ser feitos. Em particular, à Mariana Bravo, ao Hugo Corbucci, ao orientador Alfredo Goldman, ao Guilherme Silveira e aos demais profissionais da Caelum que enriqueceram o \calopsita{} com valiosas opiniões. 

No geral, há uma satisfação da equipe com o que foi desenvolvido, não só no sistema entregue, mas também no ferramental de suporte -- os outros projetos \opensource com os quais colaboramos para que se adequassem às necessidades do \calopsita{}.

E é com esse espírito de colaboração do \opensource que contamos para o sucesso do projeto. Com a facilidade de criação de \textit{plugins} e as muitas necessidades específicas de cada time, a equipe está esperançosa na multiplicação das extensões do \calopsita{}.

No mais, foi um prazer desenvolver um projeto de utilidade para a comunidade ágil e aprender as tantas diferentes tecnologias e métodos que se estudou durante a construção do \calopsita{}. E essa construção só foi possível somando-se o que foi aprendido no curso com o conhecimento adquirido no estágio.
