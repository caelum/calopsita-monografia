\documentclass[titlepage,a4paper]{article} 
\usepackage{verbatim}
\usepackage[T1]{fontenc}
\usepackage[utf8]{inputenc}
\usepackage[brazil]{babel}
\usepackage{hyperref}
\usepackage{rotating}
\usepackage{listings}
\usepackage{color}
\usepackage{float}
\usepackage{amsmath, amsthm, amssymb}
\hypersetup{colorlinks=true,%
	citecolor=red,%
	linkcolor=red,%
	urlcolor=blue,%
	pdftex}
	
\newcommand{\opensource}{\textit{open source}}
\newcommand{\software}{\textit{software}}
\newcommand{\calopsita}{Calopsita}

\title{PARTE SUBJETIVA}
\author{Cecilia Fernandes}

\begin{document}

\maketitle

Após o colegial voltado para o vestibular que cursei, passar na USP não foi uma grande felicidade, mas algo levado, pessoalmente, mais como uma transição natural. Não comemorei a entrada num excelente curso numa das melhores universidades do país.

Dessa vez, contudo, comemoro a formatura.

Faço isso porque nesses anos que passei no IME, nesses anos que foram investidos em mim, aprendi muito mais do que se poderia esperar apenas lendo a grade curricular do curso. O crescimento pessoal está incluso em cada uma das matérias, em cada dia que se passa na universidade e não pode ser ignorado nesse texto final.

Olho para o \calopsita{} com orgulho e vejo nele cada noite virada fazendo EPs, não em si pelo conteúdo do EP, mas pela persistência, pelo foco que adquiri em cada uma dessas ditas experiências acadêmicas.

Também vislumbro nele a experiência importantíssima do estágio na minha formação. A escolha por começar um estágio no terceiro ano causou um replanejamento do curso para ser concluido em cinco anos. Pode-se pensar que é um preço grande a se pagar por essa decisão, mas tenho certeza de que o lucro foi maior.

Felizmente, a empresa na qual estagiei e, mais tarde, fui efetivada, dá uma importância muito grande ao aprendizado -- afinal, é para isso que estágios servem. E, de fato, aprendi muito.

Foi graças à Caelum que conheci o mercado de trabalho brasileiro atual, aprendi sobre métodos ágeis, tecnologias atuais e, novamente, cresci no âmbito pessoal. Minha graduação seria incompleta sem esse complemento.

Nos cinco anos de bacharelado ainda tive a oportunidade de fazer um breve estágio no centro de pesquisas da IBM em Nova York. Breve, mas de enorme importância na minha vida. Nessa experiência, conheci pessoas maravilhosas e inteligentíssimas que moram no outro hemisfério.

Entre elas, destaco a professora Dilma da Silva e atrelo a ela a palavra "professora" porque aprendi muito com ela no tempo que trabalhei na IBM -- e também porque sei que é um elogio para ela. Apesar do pouco contato que tivemos desde que voltei, guardo com muito carinho tudo o que aprendi com ela, por explicações e, principalmente, por exemplo.

É uma dessas pessoas especiais que marcam a nossa vida e que se carrega com você para aqueles (muitos) momentos de decisão nos quais sua própria opinião não é o bastante para fazer uma escolha.




\section{Desafios e frustrações}



\section{Disciplinas cursadas mais relevantes}


\section{Futuro}

\section{Agradecimentos}

Devem ser citados Andrea e Jimi Xenidis, que infelizmente não poderão ler esse texto em português, mas que devem ser citados por terem dedicado tantos finais de semana a nos fazer sentir em casa em um país desconhecido.

\end{document}
