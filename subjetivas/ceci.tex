\documentclass[titlepage,a4paper]{article} 
\usepackage{verbatim}
\usepackage[T1]{fontenc}
\usepackage[utf8]{inputenc}
\usepackage[brazil]{babel}
\usepackage{hyperref}
\usepackage{rotating}
\usepackage{listings}
\usepackage{color}
\usepackage{float}
\usepackage{amsmath, amsthm, amssymb}
\hypersetup{colorlinks=true,%
	citecolor=red,%
	linkcolor=red,%
	urlcolor=blue,%
	pdftex}
	
\newcommand{\opensource}{\textit{open source}}
\newcommand{\software}{\textit{software}}
\newcommand{\calopsita}{Calopsita}

\title{PARTE SUBJETIVA}
\author{Cecilia Fernandes}

\begin{document}

\maketitle

Após o colegial voltado para o vestibular que cursei, passar na USP não foi uma grande felicidade, mas algo levado, pessoalmente, mais como uma transição natural. Não comemorei a entrada num excelente curso numa das melhores universidades do país.

Dessa vez, contudo, comemoro a formatura.

Faço isso porque nesses anos que passei no IME, nesses anos que foram investidos em mim, aprendi muito mais do que se poderia esperar apenas lendo a grade curricular do curso. O crescimento pessoal está incluso em cada uma das matérias, em cada dia que se passa na universidade e não pode ser ignorado nesse texto final.

Olho para o \calopsita{} com orgulho e vejo nele cada noite virada fazendo EPs, não em si pelo conteúdo do EP, mas pela persistência, pelo foco que adquiri em cada uma dessas ditas experiências acadêmicas.

Também vislumbro nele a experiência importantíssima do estágio na minha formação. A escolha por começar um estágio no terceiro ano causou um replanejamento do curso para ser concluido em cinco anos. Pode-se pensar que é um preço grande a se pagar por essa decisão, mas tenho certeza de que o lucro foi maior.

Felizmente, a empresa na qual estagiei e, mais tarde, fui efetivada, dá uma importância muito grande ao aprendizado -- afinal, é para isso que estágios servem. E, de fato, aprendi muito.

Foi graças à Caelum que conheci o mercado de trabalho brasileiro atual, aprendi sobre métodos ágeis, tecnologias atuais e, novamente, cresci no âmbito pessoal. Minha graduação seria incompleta sem esse complemento.

Nos cinco anos de bacharelado ainda tive a oportunidade de fazer um breve estágio no centro de pesquisas da IBM em Nova York. Breve, mas de enorme importância na minha vida. Nessa experiência, conheci pessoas maravilhosas e inteligentíssimas que moram no outro hemisfério.

Entre elas, destaco a professora Dilma da Silva e atrelo a ela a palavra "professora" porque aprendi muito com ela no tempo que trabalhei na IBM -- e também porque sei que é um elogio para ela. Apesar do pouco contato que tivemos desde que voltei, guardo com muito carinho tudo o que aprendi com ela, por explicações e, principalmente, por exemplo.

É uma dessas pessoas especiais que marcam a nossa vida e que se carrega com você para aqueles (muitos) momentos de decisão nos quais sua própria opinião não é o bastante para fazer uma escolha.

De volta à questão do curso e do projeto final, vamos às próximas seções.

\section{Desafios e frustrações}

Foram muitos os desafios do curso -- pelo menos nos momentos em que cada um deles apareceu. É engraçado, hoje, ver \textit{bixos} passando pelas exatas dificuldades que passei. É parte do aprendizado, acredito. Eis uma frase do professor Marcelo Finger que ouvi na disciplina de Banco de Dados:

\begin{quote}
	"Isso aqui é uma universidade, não é um centro de treinamento."
\end{quote}

E o significado dela é que, por mais que as coisas sejam difíceis, na faculdade é nosso trabalho aprender a aprender as mais diversas coisas sozinhos.

\subsubsection*{Desafios}

Aos desafios que pontualmente eu gostaria de deixar registrados: professores estrangeiros que não sabem falar português e não se preocupam em aprender são os que usualmente lecionam as matérias da matemática. Eis o aprendizado individual -- foi muito útil aí.

Outro desafio era, no começo, escolher as matérias mais relevantes naquele momento. Leva um certo tempo para aprendermos quais professores devemos escolher para quais matérias. Esse desafio foi superado, ainda no segundo ano, com a ajuda de veteranos gentis e prestativos que acabaram como clientes desse projeto e da amiga de mais tempo Cris Sato.

E, depois do estágio, o maior desafio foi me manter em dia com ambas as atividades e ainda manter esportes e amigos. Não foi fácil, certamente. Esse foi, ou parece, por ser mais recente, o maior dos desafios. Mas creio que eu tenha conseguido superá-lo em boa parte do tempo.

Nesse último semestre, contudo, tenho que admitir que esse desafio me superou na parte social e esportiva. Com muito o que fazer no IME, responsabilidades crescentes no trabalho e o desenvolvimento do \calopsita{}, acabei deixando de lado tudo o que era pessoal e podia ser deixado de lado.

\subsubsection*{Frustrações}

A primeira frustração da qual me lembro no BCC foram as aulas de Estatíca I. Não acreditava no quão mal um professor poderia se prestar a dar aulas. Explico. Vindo de um colégio bom, a não preparação da aula, incapacidade de resolver exercícios do livro, péssima dicção e extrema quantidade de divagações não relacionadas à materia feitas durante o curso me abismaram. 

Eu não teria coragem de apresentar uma aula como aquela. E, de fato, a vergonha pelo professor era tanta que deixei de assistir suas aulas -- preferia estudar sozinha pelo livro. Foi uma grande frustração, naquele momento, porque um professor da USP, no meu entendimento de \textit{bixete}, não poderia ser tão ruim. Infelizmente, apesar de ninguém superá-lo na minha graduação, ele não foi o único a causar desapontamento.



\section{Disciplinas cursadas mais relevantes}


\section{Futuro}

\section{Agradecimentos}

Devem ser citados Andrea e Jimi Xenidis, que infelizmente não poderão ler esse texto em português, mas que devem ser citados por terem dedicado tantos finais de semana a nos fazer sentir em casa em um país desconhecido.

Cris Sato porque é culpa dela que, entre USP e Unicamp decidi ficar no IME.

\end{document}
