\documentclass[titlepage,a4paper]{article} 
\usepackage{verbatim}
\usepackage[T1]{fontenc}
\usepackage[utf8]{inputenc}
\usepackage[brazil]{babel}
\usepackage{hyperref}
\usepackage{rotating}
\usepackage{listings}
\usepackage{color}
\usepackage{float}
\usepackage{amsmath, amsthm, amssymb}
\hypersetup{colorlinks=true,%
	citecolor=red,%
	linkcolor=red,%
	urlcolor=blue,%
	pdftex}
	
\newcommand{\opensource}{\textit{open source}}
\newcommand{\software}{\textit{software}}
\newcommand{\calopsita}{Calopsita}

\title{PARTE SUBJETIVA}
\author{Cecilia Fernandes}

\begin{document}

\maketitle

Após o colegial voltado para o vestibular que cursei, passar na USP não foi uma grande felicidade, mas algo levado, pessoalmente, mais como uma transição natural. Não comemorei a entrada num excelente curso numa das melhores universidades do país.

Dessa vez, contudo, comemoro a formatura.

Faço isso porque nesses anos que passei no IME, nesses anos que foram investidos em mim, aprendi muito mais do que se poderia esperar apenas lendo a grade curricular do curso. O crescimento pessoal está incluso em cada uma das matérias, em cada dia que se passa na universidade e não pode ser ignorado nesse texto final.

Olho para o \calopsita{} com orgulho e vejo nele cada noite virada fazendo EPs, não em si pelo conteúdo do EP, mas pela persistência, pelo foco que adquiri em cada uma dessas ditas experiências acadêmicas.

Também vislumbro nele a experiência importantíssima do estágio na minha formação. A escolha por começar um estágio no terceiro ano causou um replanejamento do curso para ser concluido em cinco anos. Pode-se pensar que é um preço grande a se pagar por essa decisão, mas tenho certeza de que o lucro foi maior.

Felizmente, a empresa na qual estagiei e, mais tarde, fui efetivada, dá uma importância muito grande ao aprendizado -- afinal, é para isso que estágios servem. E, de fato, aprendi muito.

Foi graças à Caelum que conheci o mercado de trabalho brasileiro atual, aprendi sobre métodos ágeis, tecnologias atuais e, novamente, cresci no âmbito pessoal. Minha graduação seria incompleta sem esse complemento.

Nos cinco anos de bacharelado ainda tive a oportunidade de fazer um breve estágio no centro de pesquisas da IBM em Nova York. Breve, mas de enorme importância na minha vida. Nessa experiência, conheci pessoas maravilhosas e inteligentíssimas que moram no outro hemisfério.

Entre elas, destaco a professora Dilma da Silva e atrelo a ela a palavra "professora" porque aprendi muito com ela no tempo que trabalhei na IBM -- e também porque sei que é um elogio para ela. Apesar do pouco contato que tivemos desde que voltei, guardo com muito carinho tudo o que aprendi com ela, por explicações e, principalmente, por exemplo.

É uma dessas pessoas especiais que marcam a nossa vida e que se carrega com você para aqueles (muitos) momentos de decisão nos quais sua própria opinião não é o bastante para fazer uma escolha.

De volta à questão do curso e do projeto final, vamos às próximas seções.

\section{Desafios e frustrações}

Foram muitos os desafios do curso -- pelo menos nos momentos em que cada um deles apareceu. É engraçado, hoje, ver \textit{bixos} passando pelas exatas dificuldades que passei. É parte do aprendizado, acredito. Eis uma frase do professor Marcelo Finger que ouvi na disciplina de Banco de Dados:

\begin{quote}
	"Isso aqui é uma universidade, não é um centro de treinamento."
\end{quote}

E o significado dela é que, por mais que as coisas sejam difíceis, na faculdade é nosso trabalho aprender a aprender as mais diversas coisas sozinhos.

\subsubsection*{Desafios}

Aos desafios que pontualmente eu gostaria de deixar registrados: professores estrangeiros que não sabem falar português e não se preocupam em aprender são os que usualmente lecionam as matérias da matemática. Eis o aprendizado individual -- foi muito útil aí.

Outro desafio era, no começo, escolher as matérias mais relevantes naquele momento. Leva um certo tempo para aprendermos quais professores devemos escolher para quais matérias. Esse desafio foi superado, ainda no segundo ano, com a ajuda de veteranos gentis e prestativos que acabaram como clientes desse projeto e da amiga de mais tempo Cris Sato.

E, depois do estágio, o maior desafio foi me manter em dia com ambas as atividades e ainda manter esportes e amigos. Não foi fácil, certamente. Esse foi, ou parece, por ser mais recente, o maior dos desafios. Mas creio que eu tenha conseguido superá-lo em boa parte do tempo.

Nesse último semestre, contudo, tenho que admitir que esse desafio me superou na parte social e esportiva. Com muito o que fazer no IME, responsabilidades crescentes no trabalho e o desenvolvimento do \calopsita{}, acabei deixando de lado tudo o que era pessoal e podia ser deixado de lado.

\subsubsection*{Frustrações}

No que diz respeito ao projeto, minha única frustração foi terminar o ano sem ainda ter a representação do Quadro Branco, acompanhamento mais utilizado em métodos ágeis, no \calopsita{}. Essa frustração, no entanto, foi reduzida.

Entendo que a troca do quadro branco pela criação da arquitetura de \textit{plugins} tenha realmente sido importante e agregado muito mais aos meus conhecimentos acadêmicos. Com os tantos desafios que essa arquitetura gera, passando por eles na pele, entendi melhor por que trabalhar com \textit{plugins} em outras ferramentas é tão burocrático. 

Já a primeira frustração da qual me lembro no BCC foram as aulas de Estatíca I. Não acreditava no quão mal um professor poderia se prestar a dar aulas. Explico. Vindo de um colégio bom, a não preparação da aula, incapacidade de resolver exercícios do livro, péssima dicção e extrema quantidade de divagações não relacionadas à materia feitas durante o curso me abismaram. 

Eu não teria coragem de apresentar uma aula como aquela. E, de fato, a vergonha pelo professor era tanta que deixei de assistir suas aulas -- preferia estudar sozinha pelo livro. Foi uma grande frustração, naquele momento, porque um professor da USP, no meu entendimento de \textit{bixete}, não poderia ser tão ruim. Infelizmente, apesar de ninguém superá-lo na minha graduação, ele não foi o único a causar desapontamento.

Mas provavelmente a maior frustração do BCC foi a disciplina de Engenharia de \textit{Software}. Como uma pessoa da área de Sistemas, eu tinha alguma expectativa nessa disciplina e não apenas o currículo dela se mostra extremamente desatualizado, mas também a aplicação dela pelo professor foi terrível.

Na parte técnica dessa monografia, escrevo sobre métodos ágeis e é neles que o mercado atual acredita nos últimos anos -- 2009, em particular, teve um crescimento abismal do uso de \textit{Scrum} no Brasil. E na disciplina que forma os profissionais que entrarão no mercado dalí a dois ou três anos, vemos métodos descritos em 1970.

Por importante que seja saber a história da computação, é importante que essas informações sejam passadas como História e não como se fossem a realidade atual. Nas empresas onde IMEanos vão trabalhar, quero crer que as melhores do mercado, a agilidade já está implantada ou em implantação.

Agora, falando especificamente da aplicação da disciplina, o que mais me frustrou e a todo o meu grupo foi que, apesar de termos sido o único grupo a de fato passar o semestre aplicando uma metodologia, nosso trabalho não foi reconhecido -- outro grupo que sabemos ter feito o trabalho nas duas semanas finais recebeu elogios públicos e nota dez, enquanto o nosso ficou com sete (a princípio) porque o trabalho final não estava com gráficos bonitos e outros requisitos não funcionais.

Oras, na disciplina de Engenharia de \textit{Software}, setenta porcento de aprendizado significa que não sabemos computação gráfica, enquanto quem obteve uma nota de aprendizado total e elogios passou por toda a disciplina sem aprender exatamente a engenharia de \software{} e as dificuldades de se trabalhar em um grupo grande por tempo prolongado.

Frustração, não pela nota, que era irrelevante para mim embora não para colegas que têm bolsas de iniciação, mas pelo que ela representa. Será que o entendimento do professor da matéria que ele ensinou era tão superficial que a avaliação apenas refletiu isso? Talvez porque ele faça parte do grupo de Lógica e não do de Sistemas -- não é área dele.

Frustração.

\section{Disciplinas cursadas mais relevantes}

Felizmente, ao contrário da disciplina de Engenharia de \textit{Software}, diversas matérias foram importantes para minha formação pessoal e para a construção do \calopsita{}. Em ordem cronológica, destaco as seguintes:

\begin{itemize}
	\item{\textbf{MAC122 -- Principio de Desenvolvimento de Algoritmos}

		Toda a base da computação que levaremos para a vida, independentemente de linguagem ou paradigma. Está, escondida, em cada linha de código que programamos;}
	\item{\textbf{MAC0323 -- Estruturas de Dados} 

		Outra matéria que está presente em toda linha de código que escrevemos. É muito importante para manter a estruturação do projeto e eficiência do sistema e é das poucas matérias em que nomenclatura é realmente importante: saber os nomes das estruturas de dados nos poupa um tempo precioso quando discutindo algoritmos mais complexos;}
	\item{\textbf{MAC211 -- Laboratório de Programação I} 

		Primeiro contato com repositórios de código, com \latex{} e a primeira vez que produzimos algo ``grande''. Pelo menos, grande para os parâmetros do segundo ano. Sem essas ferramentas e sem tal experiência, o \calopsita{} não poderia ser feito;}
	\item{\textbf{MAC0414 -- Linguagens Formais e Autômatos} 

		Um dos conceitos mais atuais que utilizamos no \calopsita{} foi o REST, explicado na seção de Código da parte técnica. Surpreendentemente, os conceitos de Autômatos foram bastante úteis para entender modelagens REST de problemas;}
	\item{\textbf{MAC0342 -- Laboratório de Programação Extrema}

		Participar de um projeto, assistir o desenvolvimento de outros e ver as muitas métricas variadas pelas paredes do CEC. Essa matéria foi, de fato, a inspiração para o \calopsita{}. Também foi quando tive que usar o \textit{XPlanner} pela primeira vez e sofri com sua usabilidade. O \calopsita{} não seria tão usável se o \textit{XPlanner} não fosse tão não-intuitivo;}
	\item{\textbf{MAC0441 -- Programação Orientada a Objetos}

		Foi um execelente complemento ao estágio no que diz respeito a padrões de projeto e conceitos de Orientação a Objetos. Sendo o \calopsita{} em Java, fazemos uso de todos os conceitos e de muitos padrões, como pode ser lido na parte técnica;}
\end{itemize}

\section{Futuro}

No próximo ano, pretendo seguir com o trabalho na Caelum, provavelmente focando na parte de métodos ágeis e engenharia de \software contemporânea -- métodos ágeis e tradicionais e o porquê da migração e afins. Comecei esse trabalho já esse ano, mas ainda há muito para aprender.

Pretendo também estudar a área de Interação Homem-Computador pela literatura disponível na área e colocar os conceitos em prática nos projetos em que participar. Sinto que essa é uma área em grande \textit{deficit} no mercado nacional e ela desperta meu interesse.

Estudadas ambas as áreas, tenho a intenção de investir em um curso de especialização profissional na que mais me agradar -- provavelmente cursos no exterior, talvez algum de ensino à distância.

\section{Agradecimentos}

Muitas pessoas merecem agradecimentos e não seria possível colocá-las todas aqui. Então, pontualmente, falo de algumas pessoas que fizeram uma grande diferença nesses anos de IME. 

Começo pela Cris Sato porque é culpa dela que, entre USP e Unicamp, decidi ficar no IME. O tempo que ela investiu em, durante meu colegial, me trazer para o IME, apresentar pessoas, o ambiente, a importantíssima máquina de café, me fizeram desistir de um plano já antigo de ir para Campinas.

Aos veteranos também de 2003, Mari e Hugo, pelas inúmeras e valiosas conversas sobre o curso, a vida e o que mais aparecesse. Pelos convites para escaladas (importante se manter saudável) e pela amizade.

Aos amigos mais antigos, nos quais incluo meus pais, que aguentaram muito tempo de sumiço, nervosismos e desesperos. Agradecimentos e desculpas a esses.

Devem ser citados Andrea e Jimi Xenidis, que infelizmente não poderão ler esse texto em português, mas que aparecem aqui por terem dedicado tantos finais de semana a nos fazer sentir em casa enquanto nos Estados Unidos.

Aos professores presentes, que mostraram que se importam com o curso e querem continuar melhorando-o sempre. É pelas iniciativa desses que mantemos o orgulho de estudar ou ter estudado nessa universidade. Citando o professor Marco de Canto Coral/ECA:

\begin{quote}
	``Eu cuido do meu diploma todo dia. E é função de cada um de vocês fazer isso também.''
\end{quote}

Concordo, é nossa função. E espero fazer jus a ele em cada curso que eu lecionar na Caelum, projeto \opensource{} no qual participar ou texto que publicar. Dessa vez, comemoro minha formatura.
\end{document}
