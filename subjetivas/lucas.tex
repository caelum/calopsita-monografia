\documentclass[titlepage,a4paper]{article} 
\usepackage{verbatim}
\usepackage[T1]{fontenc}
\usepackage[utf8]{inputenc}
\usepackage[brazil]{babel}
\usepackage{hyperref}
\usepackage{rotating}
\usepackage{listings}
\usepackage{color}
\usepackage{float}
\usepackage{amsmath, amsthm, amssymb}
\hypersetup{colorlinks=true,%
	citecolor=red,%
	linkcolor=red,%
	urlcolor=blue,%
	pdftex}
	
\newcommand{\opensource}{\textit{open source}}
\newcommand{\software}{\textit{software}}
\newcommand{\calopsita}{Calopsita}

\title{PARTE SUBJETIVA}
\author{Lucas Cavalcanti dos Santos}

\begin{document}

\maketitle



\section{Desafios e frustrações}

O maior desafio ao fazer o Trabalho de Formatura foi escolher um projeto bom e um grupo bom para trabalhar durante todo o ano. O projeto ``Gerenciador de Métodos Ágeis'' agradou a todos do grupo, mas precisávamos adicionar elementos mais científicos no trabalho, para que fizéssemos uma monografia de boa qualidade. Decidimos então por focar nas boas práticas de desenvolvimento de \software{} e no processo de desenvolvimento usando métodos ágeis.

Foi uma grande dificuldade manter a qualidade do código, dos testes e do resultado do projeto na medida em que ele crescia. Algumas funcionalidades nos obrigavam a fazer grandes mudanças na arquitetura do sistema, mas pela boa cobertura de testes tínhamos confiança para fazer tais mudanças sem comprometer o funcionamento do sistema.

Outra parte difícil de fazer funcionar foi a usabilidade do sistema. Queríamos um gerenciador de métodos ágeis que fosse bonito e fácil de usar, mas nenhum dos membros da equipe conhecia muito sobre \textit{design} de páginas da \textit{web}. Tivemos que aprender a usar CSS~\footnote{Cascading Style Sheet} para que o leiaute do sistema atendesse aos requisitos de usabilidade. Mais ainda, integrar o leiaute aos efeitos visuais e deixar tudo funcionando tomou muito tempo de desenvolvimento e cada vez que tínhamos que integrar uma nova funcionalidade com um efeito novo sofríamos para criar o CSS certo. Talvez se tivéssemos estudado mais sobre boas práticas de desenvolvimento de páginas da \textit{web} isso não teria acontecido.

Foi frustrante não ter dado tempo de entregar funcionalidades importantes como o \textit{Kanban} e os gráficos de \textit{burnup} e \textit{burndown}, mas pelo menos conseguimos deixar o sistema pronto para fazer essas funcionalidades facilmente.

\section{Disciplinas cursadas mais relevantes}

Muitas disciplinas contribuiram indiretamente para a construção do \calopsita{}, mas o BCC não possui muitas matérias da área da computação mais relevante para esse trabalho. O \calopsita{} tem bastante a ver com gerenciamento de equipes, métodos ágeis e engenharia de \software{}, então as disciplinas mais relevantes foram:

\begin{itemize}
	\item{\textbf{MAC0110 - Introdução à Computação} - Por ter ensinado o básico de programação}
	\item{\textbf{MAC0211 - Laboratório de Programação I} - Por ter ensinado como trabalhar em grupo e fazer um projeto}
	\item{\textbf{MAC0332 - Engenharia de Software} - Embora tenha sido a matéria mais frustrante que eu fiz no IME, essa matéria fez com que eu aprendesse a fazer um projeto mais organizado, usando métodos ágeis, com uma equipe grande. Embora o nome da matéria seja ``Engenharia de software'', ensina-se pouco disso na matéria, dando mais foco em gerenciamento de projetos usando métodos ultrapassados do que ensinando como programar de maneira eficiente e de como fazer um projeto funcionar}
	\item{\textbf{MAC0342 - Laboratório de Programação Extrema } - Por ter ensinado um método ágil, embora o projeto que eu peguei pra fazer tenha impedido que algumas das práticas de XP fossem aplicadas}
\end{itemize}

Senti falta de algumas matérias durante o curso, que tivessem mais foco em técnicas de programação como Testes de \software{} e refatoração. Uma matéria que fosse parecida com o Dojo de programação~\footnote{http://www.dojosp.org} logo no começo do curso faria com que os alunos aprendessem boas técnicas de programação que iriam ajudar em todo o restante do curso, no desenvolvimento de EPs, e na vida profissional fora do IME.

\section{Futuro}

Pretendo continuar desenvolvendo o \calopsita{} e, principalmente, continuar estudando sobre boas práticas de desenvolvimento e técnicas de programação. Mesmo gostando bastante de matemática acabei indo pra área da computação que menos tem a ver com matemática e mais com interação com pessoas, escrita e até um pouco de arte: o desenvolvimento de código e pretendo continuar nela por bastante tempo.

\end{document}
