\documentclass[titlepage,a4paper]{article} 
\usepackage{verbatim}
\usepackage[T1]{fontenc}
\usepackage[utf8]{inputenc}
\usepackage[brazil]{babel}
\usepackage{hyperref}
\usepackage{rotating}
\usepackage{listings}
\usepackage{color}
\usepackage{float}
\usepackage{amsmath, amsthm, amssymb}
\hypersetup{colorlinks=true,%
	citecolor=red,%
	linkcolor=red,%
	urlcolor=blue,%
	pdftex}
	
\newcommand{\opensource}{\textit{open source}}
\newcommand{\software}{\textit{software}}
\newcommand{\calopsita}{Calopsita}

\title{PARTE SUBJETIVA}
\author{Lucas Cavalcanti dos Santos}

\begin{document}

\maketitle

Passar na melhor faculdade do Brasil para fazer um curso de computação foi uma das melhores coisas que aconteceram na minha vida. Vim direto de uma escola pública entrando na segunda chamada do vestibular e hoje sei que o BCC foi o melhor curso que eu poderia ter feito.

Participei de muitas coisas durante a graduação: fui membro de duas gestões do CAMat, de quatro da Comissão de Trote, fui Representante Discente do CTA e estagiário da Seção de Informática do IME. Gostaria de ter feito esportes pelo IME, mas na época que eu tinha tempo eu morava longe (Cangaíba, zona leste de São Paulo) e levava mais de duas horas para ir (e para voltar) para a USP e os treinos do esporte que eu queria fazer (Handebol) eram à noite, portanto inviáveis. Mas cheguei a treinar \textit{Ultimate Frisbee} durante um tempo.

Das matérias do IME as que tive mais dificuldade foram as de Estatística. Começou em Estat I com um professor que falava muito baixo e, como eu estava no primeiro ano e não tinha descoberto ainda que assistir aulas era muito importante, acabei parando de assistir as aulas. Passei na matéria com 5.1 de média sem ter aprendido nada. Em Estat II eu não tinha a base necessária para entender a matéria, então acabei desistindo e tendo a única reprovação do curso. Essa reprovação junto com más escolhas de matérias no terceiro semestre atrasou bastante o curso.

Durante os dois primeiros anos, tive a ajuda do COSEAS com bolsas de alimentação e de moradia, que eu acabei usando para pagar o transporte da minha casa até o IME. No fim do segundo ano tive que procurar um estágio, e acabei entrando para a Seção de Informática do IME. Infelizmente, por causa do estágio, não consegui recuperar o atraso do curso, e então me programei para terminar o curso em 5 anos.

No começo do terceiro ano fui indicado para fazer estágio na Caelum. Foi o melhor estágio que eu poderia ter conseguido, em uma empresa ótima para trabalhar que sem dúvida supriu todas as necessidades para o mercado de trabalho que faltam no BCC. Tive o suporte de ótimos profissionais da área que fizeram com que eu ficasse atualizado e me tornasse um bom programador e um bom profissional. A Caelum me apoiou nas minhas decisões sobre o curso e ainda me apoiou no desenvolvimento do \calopsita{}, patrocinando horas de trabalho e cedendo infra-estrutura para que o projeto acontecesse.

Enfim, agradeço ao Prof. Dr. Carlos E. Ferreira por me indicar para a Caelum (além de ser um dos melhores professores que eu tive no IME), ao Paulo Silveira, ao Guilherme Silveira e à Caelum em geral por me ensinarem a maior parte do que eu sei sobre programação, ao Prof. Dr. Alfredo Goldman por ter aceitado orientar esse trabalho de formatura, à Mariana Bravo, ao Hugo Corbucci e ao Guilherme Silveira por aceitarem ser clientes do projeto e ao Cauê Guerra e à Cecilia Fernandes por desenvolverem o \calopsita{} junto comigo.

\section{Desafios e frustrações}

O maior desafio ao fazer o Trabalho de Formatura foi escolher um projeto bom e um grupo bom para trabalhar durante todo o ano. O projeto ``Gerenciador de Métodos Ágeis'' agradou a todos do grupo, mas precisávamos adicionar elementos mais científicos no trabalho, para que fizéssemos uma monografia de boa qualidade. Decidimos então por focar nas boas práticas de desenvolvimento de \software{} e no processo de desenvolvimento usando métodos ágeis.

Foi uma grande dificuldade manter a qualidade do código, dos testes e do resultado do projeto na medida em que ele crescia. Algumas funcionalidades nos obrigavam a fazer grandes mudanças na arquitetura do sistema, mas pela boa cobertura de testes tínhamos confiança para fazer tais mudanças sem comprometer o funcionamento do sistema.

Outra parte difícil de fazer funcionar foi a usabilidade do sistema. Queríamos um gerenciador de métodos ágeis que fosse bonito e fácil de usar, mas nenhum dos membros da equipe conhecia muito sobre \textit{design} de páginas da \textit{web}. Tivemos que aprender a usar CSS~\footnote{Cascading Style Sheet} para que o leiaute do sistema atendesse aos requisitos de usabilidade. Mais ainda, integrar o leiaute aos efeitos visuais e deixar tudo funcionando tomou muito tempo de desenvolvimento e cada vez que tínhamos que integrar uma nova funcionalidade com um efeito novo sofríamos para criar o CSS certo. Talvez se tivéssemos estudado mais sobre boas práticas de desenvolvimento de páginas da \textit{web} isso não teria acontecido.

Foi frustrante não ter dado tempo de entregar funcionalidades importantes como o \textit{Kanban} e os gráficos de \textit{burnup} e \textit{burndown}, mas pelo menos conseguimos deixar o sistema pronto para fazer essas funcionalidades facilmente.

\section{Disciplinas cursadas mais relevantes}

Muitas disciplinas contribuiram indiretamente para a construção do \calopsita{}, mas o BCC não possui muitas matérias da área da computação mais relevante para esse trabalho. O \calopsita{} tem bastante a ver com gerenciamento de equipes, métodos ágeis e engenharia de \software{}, então as disciplinas mais relevantes foram:

\begin{itemize}
	\item{\textbf{MAC0110 -- Introdução à Computação e MAC122 -- Principio de Desenvolvimento de Algoritmos} - Por ter ensinado o básico de programação}
	\item{\textbf{MAC0211 -- Laboratório de Programação I} - Por ter ensinado como trabalhar em grupo e fazer um projeto}
	\item{\textbf{MAC0332 -- Engenharia de Software} - Embora tenha sido a matéria mais frustrante que eu fiz no IME, essa matéria fez com que eu aprendesse a fazer um projeto mais organizado, usando métodos ágeis, com uma equipe grande. Embora o nome da matéria seja ``Engenharia de software'', ensina-se pouco disso na matéria, dando mais foco em gerenciamento de projetos usando métodos ultrapassados do que ensinando como programar de maneira eficiente e de como fazer um projeto funcionar. Além disso, o método de avaliação não foi nem um pouco compatível com o proposto pela matéria, nem mesmo com o que foi combinado no começo do semestre.}
	\item{\textbf{MAC0342 -- Laboratório de Programação Extrema } - Por ter ensinado um método ágil em específico, embora o projeto que eu peguei pra fazer tenha impedido que algumas das práticas de XP fossem aplicadas}
\end{itemize}

Senti falta de algumas matérias durante o curso, que tivessem mais foco em técnicas de programação como Testes de \software{} e refatoração. Uma matéria que fosse parecida com o Dojo de programação~\footnote{http://www.dojosp.org} logo no começo do curso faria com que os alunos aprendessem boas técnicas de programação que iriam ajudar em todo o restante do curso, no desenvolvimento de EPs, e na vida profissional fora do IME.

\section{Futuro}

Pretendo continuar desenvolvendo o \calopsita{} e, principalmente, continuar estudando sobre boas práticas de desenvolvimento e técnicas de programação. Mesmo gostando bastante de matemática acabei indo pra área da computação que menos tem a ver com matemática e mais com interação com pessoas, escrita e até um pouco de arte: o desenvolvimento de código e pretendo continuar nela por bastante tempo.

\end{document}
