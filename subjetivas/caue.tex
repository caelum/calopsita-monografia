\documentclass[titlepage,a4paper]{article} 
\usepackage{verbatim}
\usepackage[T1]{fontenc}
\usepackage[utf8]{inputenc}
\usepackage[brazil]{babel}
\usepackage{hyperref}
\usepackage{rotating}
\usepackage{listings}
\usepackage{color}
\usepackage{float}
\usepackage{amsmath, amsthm, amssymb}
\hypersetup{colorlinks=true,%
	citecolor=red,%
	linkcolor=red,%
	urlcolor=blue,%
	pdftex}
	
\newcommand{\opensource}{\textit{open source}}
\newcommand{\software}{\textit{software}}
\newcommand{\calopsita}{Calopsita}

\title{PARTE SUBJETIVA}
\author{Cauê Haucke Porta Guerra}

\begin{document}

\maketitle

\tableofcontents
\newpage
\section{Desafios e frustrações}
desafios:
- convencer o orientador, antes de ele de fato aceitar essa funcao, de que o projeto era legal, de que tinha diferenciais
em relacao aos demais, que teria pesquisa em cima, que daria pra extrair um tcc de verdade, uma das preocupacoes


frustrações
- 


\section{Disciplinas cursadas mais relevantes}
mac110



\section{Futuro}


\end{document}
