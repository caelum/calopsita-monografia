\documentclass[titlepage,a4paper]{article} 
\usepackage{verbatim}
\usepackage[T1]{fontenc}
\usepackage[utf8]{inputenc}
\usepackage[brazil]{babel}
\usepackage{hyperref}
\usepackage{rotating}
\usepackage{listings}
\usepackage{color}
\usepackage{float}
\usepackage{amsmath, amsthm, amssymb}
\hypersetup{colorlinks=true,%
	citecolor=red,%
	linkcolor=red,%
	urlcolor=blue,%
	pdftex}
	
\newcommand{\opensource}{\textit{open source}}
\newcommand{\software}{\textit{software}}
\newcommand{\calopsita}{Calopsita}

\title{PARTE SUBJETIVA}
\author{Cauê Haucke Porta Guerra}

\begin{document}

\maketitle


\newpage
\section{Desafios e frustrações}
desafios:
- convencer o orientador, antes de ele de fato aceitar essa funcao, de que o projeto era legal, de que tinha diferenciais
em relacao aos demais, que teria pesquisa em cima, que daria pra extrair um tcc de verdade, uma das preocupacoes


frustrações
- 


\section{Disciplinas cursadas mais relevantes}

Eu diria que todas as disciplinas cursadas foram fundamentais na realização desse trabalho e na formação do profissional que sou hoje. Mesmo o \calopsita{} sendo um sistema, as matérias mais teóricas e matemáticas tiveram sua importância na formação da base necessária e no formalismo exigidos de um cientista da computação. No entanto, as que influenciaram mais diretamente na realização desse trabalho foram:

\begin{itemize}
	\item{\textbf{MAC0110 - Introdução à Computação} - importante por ser o primeiro contato dos alunos com computação. Serve de base para todo o resto.}
	\item{\textbf{MAC0122 - Princípios de Desenvolvimento de Algoritmos} - junto com MAC0110, forma a base de todo o conhecimento que será adquirido durante o curso.}
	\item{\textbf{MAC0211 - Laboratório de Programação I} - a importância dessa disciplina pra mim foi o aprendizado de LaTex e sistemas de controles de versão, além de ser a primeira disciplina onde tivemos de trabalhar em um projeto de duração um pouco maior.}
	\item{\textbf{MAC0242 - Laboratório de Programação II} - }
	\item{\textbf{MAC0332 - Engenharia de Software} - }
	\item{\textbf{MAC0342 - Laboratório de Programação Extrema} - }
	\item{\textbf{MAC0441 - Programação Orientada a Objetos} - }
\end{itemize}

Ainda gostaria de citar o apoio do IME a diversas atividades extra-curriculares, em especial ao patrocínio de caravanas ao FISL (Forum Internacional de Software Livre), que propiciaram meu primeiro contato com a comunidade de desenvolvimento de \software{}.

\section{Futuro}


\end{document}
