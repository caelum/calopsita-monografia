\section{Funcionalidades}

No \calopsita{}, também por sua arquitetura de \textit{plugins}, as funcionalidades estão separadas em duas grandes partes: o núcleo e os \textit{plugins}. A primeira, contém apenas partes diretamente relacionadas com desenvolvimento ágil, no sentido mais amplo e irrestrito do termo -- sem interferência de metodologias, seus métodos e métricas. Essas partes variáveis de cada projeto ou dependente de metodologia são deixadas para os \textit{plugins}, que dão ao sistema uma melhor adaptabilidade.

\subsection{Calopsita \textit{Core}}

As funcionalidades que fazem parte do núcleo do \calopsita{} consistem da criação e administração de usuários, projetos, cartões e iterações. Parece ser um núcleo minimal e essa é a intenção, contudo a proposta dos cartões é um tanto diferente e precisa ser explicada.

\subsubsection*{Cartões}

Diferente de outros sistemas com o mesmo propósito, o \calopsita{} não possui o conceito de histórias, mas apenas de cartões. Isso foi feito pensando em trazer maior grau de customização para os usuários. Cada cartão pode ter subcartões e o que define a funcionalidade desse cartão é o conjunto de \textit{gadgets} que ele possui. 

A vantagem é que pode-se criar uma hierarquia, tão profunda quanto se desejar, para organizar tudo o que há para ser feito em um projeto. Isso também permite que o projeto possa ser visto no nível de detalhe mais apropriado pra cada envolvido no projeto, seja ele gerente, desenvolvedor ou cliente. 

\subsubsection*{Tipos de Cartões}

Um tipo de cartão é, para o \calopsita{}, um agrupamento de \textit{gadgets} que definem o comportamento de um determinado cartão. Perceba que a noção é puramente semântica, já que os \textit{gadgets} podem ser habilitados e desabilitados individualmente, por cartão.


*****************
Falta coisa aqui... muita, btw
*****************

 

\subsection{Calopsita Plugins}

Como explicado anteriormente, o \calopsita{} possuí um núcleo com as funcionalidades essenciais e as demais serão fornecidas através de \textit{plugins}. Temos em nosso \textit{backlog}, diversos plugins para serem implementados e que já viriam na instalação padrão do \calopsita{}. Segue abaixo uma breve descrição de cada um deles:

\begin{itemize}
	\item{Priorização: permite seja atribuída uma prioridade a um determinado cartão através de uma interface baseada em \textit{drag'n drop}. Esse plugin já está implementado.}
	\item{Planejamento: permite que cartões sejam adicionadas ou removidas de uma determinada iteração. Também baseado em \textit{drag'n drop} e já está implementado.}
	\item{Gráfico burn-up: a idéia desse plugin é que uma nova página contendo o gráfico burnup de uma determinada iteração sejá criada.}
	\item{Gráfico burn-down: a idéia desse plugin é que uma nova página contendo o gráfico burndown de uma determinada iteração sejá criada.}
	\item{Marcação de horas: esse plugin permitirá que a pessoa que trabalhou na conclusão de determinado cartão possa marcar o número de horas trabalhadas.}
	\item{Estimativas: possibilitará que um cartão tenha sua dificuldade estimada em pontos.}
	\item{Personas: criará uma área especial para a definição de personas. Essas personas poderão ser utilizadas mais facilmente na criação de novos cartões.}
	\item{Integração com GitHub: permitírá que comentários feitos no \textit{commit} do projeto no GitHub consigam mover cartões dentro de uma determinada iteração e mudar seu status.}
\end{itemize}


\textbf{Priorização}
\textbf{Gráficos}
\textbf{Estimativas}
\textbf{Marcação de horas}
\textbf{Customizável}
\textbf{Templates de metodologias}
\textbf{Templates de cartões}
\textbf{Personas}

