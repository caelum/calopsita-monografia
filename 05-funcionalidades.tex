\section{Funcionalidades}

No \calopsita, também por sua arquitetura de plugins, as funcionalidades estão separadas em duas grandes partes: o núcleo e os plugins. A primeira, contém apenas partes diretamente relacionadas com desenvolvimento ágil, no sentido mais amplo e irrestrito do termo -- sem interferência de metodologias, seus métodos e métricas. Essas partes variáveis de cada projeto ou dependente de metodologia são deixadas para os plugins, que dão ao sistema uma melhor adaptabilidade.

\subsection{Calopsita Core}

As funcionalidades que fazem parte do núcleo do \calopsita consistem da criação e administração de usuários, projetos, cartões e iterações. Parece ser um núcleo minimal e essa é a intenção, contudo a proposta dos cartões é um tanto diferente e precisa ser explicada.

\subsubsection*{Cartões}

Diferente de outros sistemas com o mesmo propósito, o \calopsita não possui o conceito de estórias, mas apenas de cartões. Isso foi feito pensando em trazer maior grau de customização para os usuários. Cada cartão pode ter subcartões e o que define a funcionalidade desse cartão é o conjunto de \textit{gadgets} que ele possui. 

A vantagem é que pode-se criar uma hierarquia, tão profunda quanto se desejar, para organizar tudo o que há para ser feito em um projeto. Isso também permite que o projeto possa ser visto no nível de detalhe mais apropriado pra cada envolvido no projeto, seja ele gerente, desenvolvedor ou cliente. 

\subsubsection*{Tipos de Cartões}

Um tipo de cartão é, para o \calopsita, um agrupamento de \textit{gadgets} que definem o comportamento de um determinado cartão. Perceba que a noção é puramente semântica, já que os \textit{gadgets} podem ser habilitados e desabilitados individualmente, por cartão.


*****************
Falta coisa aqui... muita, btw
*****************

 

\subsection{Calopsita Plugins}

\textbf{Priorização}
\textbf{Gráficos}
\textbf{Estimativas}
\textbf{Marcação de horas}
\textbf{Customizável}
\textbf{Templates de metodologias}
\textbf{Templates de cartões}
\textbf{Personas}


burnup
burn down


