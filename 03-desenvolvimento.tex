\section{Desenvolvimento}

Já desde o início de janeiro, a proposta inicial do Calopsita foi montada. Durante o mês de janeiro, estudamos tecnologias que poderiam ser interessantes no desenvolvimento do projeto e conseguimos o apoio do professor Alfredo Goldman, que aceitou nos orientar, da Caelum, que nos cedeu horas do trabalho e dos mestrandos Mariana Bravo e Hugo Corbucci, nossos clientes eleitos.

Também desde então temos o grupo de discussões~\footnote{http://groups.google.com/group/calopsita-development} e um repositório no GitHub para o projeto. Esse repositório foi movido durante o desenvolvimento, mas seu histórico completo se manteve. Abaixo, um gráfico de número de linhas enviadas ao repositório no decorrer do ano ilustra a distribuição do trabalho de código durante o ano.

\includegraphics[scale=0.4]{images/impacto.png}

Esses dados foram colhidos do gráfico de impacto do GitHub e as divisões de linhas por desenvolvedor foram removidas porque, como adotamos programação pareada na maioria do tempo, ele não necessariamente representa a cota de cada desenvolvedor no projeto.

Nota-se dessa imagem um pico de linhas de código enviadas ao repositório no mês de maio, mês seguinte a começarmos a gerenciar o próprio Calopsita usando a parte que já estava pronta do projeto. Contudo, houve trabalho de janeiro até o momento atual, com uma considerável queda em setembro para que focássemos nessa monografia.

Desde janeiro, non-stop
Clientes reais e distribuidos
Muito pareamento
Kanban + Calopsita
Usamos XP e ?????
TDD e Integração Contínua

