\section{Desenvolvimento do VRaptor3}
\label{app:vraptor}

O VRaptor~\footnote{http://vraptor.caelum.com.br} é um projeto \opensource{} desenvolvido pela Caelum, inicialmente idealizado pelos irmãos Paulo e Guilherme Silveira em 2004. É um \textit{framework web} orientado a ações, que segue o padrão MVC~\cite{mvc} auxiliando principalmente a parte dos Controladores. Surgiu como uma alternativa ao \textit{framework} mais usado da época, o Struts~\footnote{http://struts.apache.org/}, que possui muitos problemas como configurações excessivas, alto acoplamento e incentivo a escrever classes grandes que têm muitas responsabilidades.

Na versão 2~\footnote{a versão 1 não chegou a ser lançada oficialmente}, o VRaptor resolve alguns dos problemas citados acima favorecendo Convenções sobre Configurações e classes simples java~\cite{pojo}, tornando o desenvolvimento mais simples e rápido. Após dois anos de desenvolvimento, o VRaptor 2 começou a acumular problemas, e algumas das práticas usadas se mostraram ruins com o passar do tempo e sistemas grandes começavam a ficar de difícil manutenção. Por causa disso, no final de 2008 foi decidido pela reformulação total do \textit{framework}, removendo as idéias consideradas ruins, e acrescentando novas idéias como Injeção de Dependências~\cite{di} e serviços web RESTful~\cite{rest}. A versão 3 então começou a ser desenvolvida, mas precisava de aplicações que pudessem testá-la, identificando problemas, e sugerindo novas funcionalidades.

O \calopsita{} começou a ser desenvolvido com o VRaptor 2, mas foi migrado facilmente para o VRaptor3 em julho desse ano, três meses antes do seu lançamento oficial. Logo no início, foi possível identificar vários problemas (\textit{bugs}) que foram prontamente corrigidos. O VRaptor é bastante modularizado e usa injeção de dependências para juntar suas partes, assim todos os componentes recebem como dependência interfaces internas, e o contêiner de injeção de dependências decide qual implementação vai ser usada. Por causa disso, é possível criar outra implementação de alguma das interfaces do VRaptor, e essa nova implementação será usada ao invés da padrão. Assim foi possível contornar todos os \textit{bugs} encontrados antes que eles fossem corrigidos de uma forma fácil, e também modificar alguns comportamentos que não eram convenientes para o \calopsita.

Muitas funcionalidades do VRaptor foram sugeridas pelo \calopsita{} durante o seu desenvolvimento, pois em várias situações a implementação de uma funcionalidade do \calopsita{} requeria algo que não estava implementado ainda no VRaptor. Por exemplo, as URIs mais representativas usadas no \calopsita{} contém parâmetros que precisam ser extraídos e passados para os métodos java que vão tratá-las, como a URI ``/projects/10/iterations'' que contém o número 10 correspondendo ao identificador de um projeto.

Usar o VRaptor antes dele ser lançado oficialmente trouxe grandes benefícios aos dois projetos: ao \calopsita{} por ter seu desenvolvimento facilitado, e ao VRaptor pelas sugestões de funcionalidades e identificação de \textit{bugs}.
