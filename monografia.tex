\documentclass[titlepage,a4paper]{article} 
\usepackage{verbatim}
\usepackage[T1]{fontenc}
\usepackage[utf8]{inputenc}
\usepackage[brazil]{babel}
\usepackage{hyperref}
\usepackage{rotating}
\usepackage{amsmath, amsthm, amssymb}
\hypersetup{colorlinks=true,%
	citecolor=red,%
	linkcolor=red,%
	urlcolor=blue,%
	pdftex}

\title{Calopsita: Um sistema gerenciador de projetos que utilizam metodologias ágeis}
\author{Cauê Haucke Porta Guerra\\Cecilia Fernandes\\Lucas Cavalcanti dos Santos\\ \\Orientador: Prof. Dr. Alfredo Goldman}

\begin{document}

\maketitle

\begin{description} 
\item{\textbf{Alunos:}\\Cauê Haucke Porta Guerra\\Cecilia Fernandes\\Lucas Cavalcanti dos Santos}
\item{\textbf{Supervisor:}\\Prof. Dr. Alfredo Goldman}
\item{\textbf{Colaboradores:}\\Mariana V. Bravo\\Hugo Corbucci\\Paulo E. de Azevedo Silveira\\Guilherme de Azevedo Silveira}
\end{description}

\section{Introdução}
Calopsita é um projeto que nasceu da necessidade de seus idealizadores, que trabalham na mesma empresa, em gerenciar projetos ágeis que utilizavam diferentes métodos e com o diferencial de seu time ser distribuído. Essa necessidade ficou evidente quando alguns de nós estávamos alocados em algum projeto de consultoria e esse projeto era desenvolvido dentro da própria empresa e o cliente ainda tinha a necessidade de acompanhar a evolução do mesmo de uma maneira mais dinâmica.

Aproveitando que dois amigos nossos, Hugo Corbucci e Mariana Bravo, estavam fazendo mestrado nessa área, resolvemos convidá-los para serem clientes do projeto, convite esse que foi aceito prontamente. Além deles, contamos com o apoio do nosso orientador, que sempre esteve envolvido e acompanhava o andamento do projeto, e com alguns amigos da Caelum, que foram de fundamental importância nas discussões sobre a arquitetura do sistema, e que depois tornaram-se clientes e passaram a pedir por novas funcionalidades e reportar bugs. Hoje, o Calopsita já é usado para gerenciar seu próprio desenvolvimento (que pode ser acessado aqui:) e o desenvolvimento de vários outros projetos internos da Caelum.

\section{Motivação e público alvo}

qualquer metodologia agil
time distribuido
projetos open source
agilistas

\section{Desenvolvimento}
Java
Hibernate
Plugins
VRaptor 2/3 (migração)

ActiveRecord
post do Lucas aqui, ja q n foi publicado

BDD
citar nosso post de DDD, colocar exemplos

\section{Código}

onde ta
descricao tecnica (cuidado pra n ficar chato)

\section{Funcionalidades}

O Calopsita é feito utilizando uma arquitetura de plugins. Ou seja, a aplicação tem um núcleo, contendo a parte básica do sistema e todo o resto é feito ou configurado com a criação de plugins, dando um grande poder de customização.
Entre as funcionalidades que fazem parte do núcleo estão a criação e administração de usuários,
projetos, cartões, iterações e ações para log.

O Calopsita, ao contrário de outros sistemas com o mesmo propósito, não possui o conceito de histórias, apenas de cartões. Isso foi feito pensando em trazer maior grau de customização para os usuários, uma vez que cartões podem ter subcartões. Essa funcionalidade permite uma hierarquia tão profunda quanto se deseje, e permite que o projeto seja visto no nível de detalhe mais apropriado pra cada envolvido no projeto, seja um gerente, desenvolvedor, product owner, etc.

A arquitetura de plugins

Esses são os plugins que implementamos:
priorizacao
burnup
burn down
marcar horas
estimativa
templates de metodologias
templates de cartão

\section{Integração}
pretendemos ainda integrar com github, ou outros meios, para atualizacao dinamica de cartoes
treco de desenhos do lipe (sketch blabla)


\section{Visão dos clientes e comparativos}

clientes: mari, hugo, caelum

Metodologias

XP
Lean
Scrum
Crystal


\begin{sidewaystable}
	\begin{tabular}{|l|l|l|l|l|l|l|l}
		\hline
		\multicolumn{8}{|c|}{Aplicações similares} \\
		\hline
		 & PivotalTracker & Scrumy & ScrumNinja & Scrumd & VersionOne & BlueSoft & Mingle \\
		Gratuito & X & & & & & & \\
		OpenSource & - & & & & & & \\
		Graficos & X & & & & & & \\
		Estimativas & X & & & & & & \\
		Priorizacao & X & & & & & & \\
		Marcacao Horas & - & & & & & & \\
		Customizavel & & & & & & & \\
		Templates Metodologias & & & & & & & \\
		Templates Cartão & & & & & & & \\
		Personas & & & & & & & \\
		\hline
	\end{tabular}
\end{sidewaystable}

\section{Conclusão}

\section{Bibliografia}

\end{document}