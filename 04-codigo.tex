\section{Código}
Todo o código do Calopsita está hospedado no GitHub, local onde também pode ser visualizado todo o histórico e atividades do projeto desde sua concepção. http://github.com/caelum/calopsita/

Antes de iniciar o desenvolvimento do Calopsita, foi discutido quais seriam as escolhas de linguagens e frameworks a serem utilizados. As opções eram: Ruby on Rails ou Java com VRaptor. Como a maioria da equipe era mais familiar com Java do que com Ruby, a escolha foi pelo Java com VRaptor.

\subsection{BDD}
Desde o começo resolvemos adotar boas práticas de desenvolvimento, como {\it Behavior Driven Development} (BDD), 
refatoração constante e programação pareada. Principalmente para usar BDD precisávamos de ferramentas que nos 
ajudassem a escrever testes mais expressivos e legíveis. No mundo do Ruby on Rails essa prática já está bastante 
difundida, então existem várias ferramentas de teste como o RUnit e o Cucumber, além da linguagem ser dinâmica e 
possibilitar a escrita de DSLs e interfaces fluentes de uma maneira bastante direta. Em Java a prática de BDD ainda 
não está muito forte, então não existem ferramentas muito avançadas, além da linguagem ser estática e burocrática, 
que torna o desenvolvimento de tais ferramentas muito mais difícil. Mesmo assim, foram pesquisadas as alternativas:

-JBehave - que possibilita a escrita de arquivos texto com a descrição das funcionalidades, e uma classe Java 
correspondente a esse arquivo. Foi descartado por causa da sua sintaxe pouco produtiva e de algumas limitações, 
como a impossibilidade de compartilhar passos entre features ****melhorar essa frase****
-Cucumber + JRuby - poderíamos também usar o próprio Cucumber do Ruby, usando o JRuby para poder integrar código 
Java aos testes. Foi descartada por causa da complexidade da integração, e por causa da mistura de linguagens.

** o que mais mesmo? **

-JUnit + Bom senso - Não usar nenhuma ferramenta específica, apenas JUnit para rodar os testes, e o bom senso da 
equipe para escrever testes legíveis e expressivos.

Acabamos por optar pela última opção, que era bem mais simples e poderíamos criar uma nova forma de desenvolver 
testes em Java. A nossa solução para testes foi publicada no blog da Caelum: 
http://blog.caelum.com.br/2009/02/28/behavior-driven-development-com-junit/. Essa solução é uma arquitetura de 
testes que possibilita a escrita de testes de aceitação, em Java, quase em linguagem natural. Um exemplo de teste 
escrito nessa arquitetura, tirado do código do calopsita:

TODO -> colorir isso
/**
 * <b>In order to</b> plan what has to be done <br>
 * <b>As a</b> project client <br>
 * <b>I want to</b> create and edit cards (with name and description) <br>
 *
 */
public class CreateACardStory extends DefaultStory {

	@Test
	public void cardCreation() throws Exception {
		given.thereIsAnUserNamed("David").and()
			.thereIsAProjectNamed("Papyrus").ownedBy("David").and()
			.iAmLoggedInAs("David");

		when.iOpenProjectPageOf("Papyrus").and()
		    .iOpenCardsPage().and()
			.iAddTheCard("Incidents").withDescription("create and update an incident").and()
			.iOpenCardsPage();
		then.theCard("Incidents").appearsOnList();
	}
}

que se removermos os ruidos sintáticos e os Camel cases, ficaria:

{\bf In order to} plan what has to be done
{\bf As a} project client
{\bf I want to} create and edit cards (with name and description)

Create a card story:

card creation:
	
		given there is an user Named "David" and
			there is a project named "Papyrus" owned by "David" and
			i am logged in as "David"

		when I open project page of "Papyrus" and
		  I open cards page and
			I add the card "Incidents" with description "create and update an incident" and
			I open cards page
			
		then the card "Incidents" appears on list

Isso é bastante próximo da linguagem natural, em inglês, e possibilita a leitura fácil até para leigos.

Essa arquitetura de testes foi adotada apenas para testes de aceitação, que eram escritos a cada solicitação de
funcionalidade. Para os testes unitários, decidimos apenas usar de refatoração, em especial a 
*****Extract Method - colocar a citação para o livro do Martin Fowler****, possibilitando a escrita de testes bastante
legíveis, mas com um pouco mais de ruidos sintáticos do Java. Um exemplo de teste unitário, tirado do código do calopsita:

TODO- colorir isso

public class IterationTest {
	@Test
	public void addingACardInAnIteration() throws Exception {
		Iteration iteration = givenAnIteration();
		Card card = givenACard();

		shouldUpdateTheCard(card);

		whenIAddTheCardToIteration(card, iteration);

		assertThat(card.getIteration(), is(iteration));
		mockery.assertIsSatisfied();
	}
}

Mesmo com mais ruído sintático é possível entender facilmente o intuito do teste:

Adicionando um cartão em uma iteração:
  dado uma Iteração e
  dado um cartão
  
  deve atualizar o cartão
  
  quando eu adicionar o cartão na iteração
  
  no final, garanta que a iteração do cartão é a iteração acima
  

Dessa forma, a manutenção dos testes fica muito fácil, e pode ser feita facilmente por qualquer pessoa, pois o teste deixa bem claro que está fazendo.


\section{VRaptor}

Como a escolha do Calopsita foi Java com VRaptor, começamos a desenvolver com o VRaptor na versão 2.6. 
Ao mesmo tempo, a versão nova do VRaptor, a 3.0, começou a ser desenvolvida, e por volta de julho já tinha 
uma versão alfa funcional. O VRaptor 3.0 possibilita o desenvolvimento de aplicações RESTful ****citação**** 
e o uso massivo de injeção de dependências ******citação********. Dessa forma, era possível adicionar mais 
características e boas práticas ao Calopsita. O uso de uma interface web RESTful traz várias vantagens para 
uma aplicação web. Usando os verbos HTTP do jeito certo, você consegue aproveitar a semântica da web para a 
sua aplicação, além de conseguir aproveitar recursos dos servidores, como caching. Além disso, a aplicação 
se torna automaticamente um web-service, facilitando a integração com outros sistemas.

Por volta de julho, o calopsita foi totalmente migrado para VRaptor 3, deixando o desenvolvimento mais produtivo. 
Nessa época, o VRaptor ainda não tinha uma versão estável, mas a maneira com o que ele foi feito possibilitava a fácil
personalização e resolução dos problemas que existiam nele. A versão final só saiu no começo de outubro, então durante 
julho e outubro o calopsita auxiliou no desenvolvimento do VRaptor 3, que é um projeto \opensource desenvolvido pela Caelum.

Usar injeção de dependências faz com que a aplicação fique naturalmente mais testável e menos acoplada. Isso aliado a
Factories *** citar o design pattern ***, e o uso de interfaces ao invés de implementações ***** citar o design 
pattern/effective java ****** fez com que as classes do calopsita ficassem bem testáveis unitariamente, possibilitando 
uma cobertura por testes de mais de 85\% (que é bem difícil de se conseguir em projetos java). Isso também auxiliou bastante
o desenvolvimento da arquitetura em plugins do calopsita.

\section{ActiveRecord}



O Calopsita está sendo desenvolvido com a linguagem Java, utilizando ainda Hibernate e javascript. Como framework MVC começamos utilizando o VRaptor 2 e durante o desenvolvimento do VRaptor 3 resolvemos fazer a migração. Muita coisa que precisavamos ainda não contava com o suporte do VRaptor 3, portanto nossa adoção ajudou no desenvolvimento desse framework. Ainda foi a adoção desse framework que nos permitiu a utilização de url's amigáveis, restful, etc. Sua principal vantagem é: blabla




ActiveRecord + ferreira
http://blog.objectmentor.com/articles/2007/11/02/active-record-vs-objects
post do Lucas aqui, ja q n foi publicado

Ainda SeleniumDSL (HtmlUnit x Selenium)

BDD
citar nosso post de BDD, colocar exemplos
http://blog.caelum.com.br/2009/02/28/behavior-driven-development-com-junit/

Desenvolvimento paralelo do VRaptor, um ajudou o outro
lembrar que DI eh design pattern

descricao tecnica (cuidado pra n ficar chato)


