\section{Entrevista com os clientes}
\label{entrevista}

Para conseguirmos extrair a visão dos clientes, fizemos a seguinte entrevista:\\

\textbf{\calopsita{}}: O que vc esperava do projeto no inicio do ano?\\

\textbf{Hugo Corbucci}: Quando começamos a conversar, eu imaginei um mundo ideal no qual, ao final do ano, poderíamos escrever cartões de história, colocá-los em iterações e marcar neles o trabalho realizado. Com esses dados, poderíamos ver o progresso do trabalho como um todo olhando para a iteração. De uma certa forma, esperava conseguir ter as principais funcionalidades que usamos no \textit{XPlanner} mas em um sistema mais novo com uma interface bem melhor. Acho que isso descreve bem minha primeira ``visão'' do projeto.\\

\textbf{Guilherme Silveira}: Poder substituir a lousa por um projetor... um sonho bem alto.\\

\textbf{Mariana Bravo}: Poder gerenciar minimamente um projeto, ou seja: criar o projeto, adicionar pessoas, criar cartões, planejar iterações.
Eu não tinha uma visão muito definida do que eu esperava que estivesse pronto, mas tinha uma visão das coisas que eram e são importantes pra mim. Por exemplo, colocar horas nas tarefas é importante pra mim, mas eu não esperava que isso estivesse pronto no final do ano...\\

\textbf{\calopsita{}}: Como cliente, o que você achou do processo de desenvolvimento do \calopsita{}? Do que sentiu falta? O que achou mais legal?\\

\textbf{Hugo Corbucci}: O processo de desenvolvimento foi bem legal. Gostei bastante do ritmo que tivemos até agosto. Depois disso, tanto por ausência nossa quanto de vocês, ficou um pouco mais parado. Senti falta de estarmos mais próximos nos períodos de programação mesmo. O ambiente de deploy contínuo foi muito bom mas tivemos alguns problemas críticos que tardaram a ser resolvidos. A ideia de montar as \textit{personas} para conseguirmos conversar melhor foi muito legal.\\

\textbf{Guilherme Silveira}: Fui um cliente ``secundario'' - eu era um objetivo mas não tão importante quanto os clientes mais próximos, por isso só cheguei a influenciar funcionalidades mais pra frente. \\

\textbf{Mariana Bravo}: Achei muito legal!
Gostei muito de fazer e pensar em \textit{personas}. Gostei de gerenciar o próprio \calopsita{} no \calopsita{}: apesar dos problemas, isso nos ajudou a ter uma visão e uma noção muito clara das coisas mais importantes para o projeto.
Senti falta, principalmente no começo que não tínhamos o \calopsita{}, de saber o que a equipe estava fazendo. Algumas vezes vinha a pergunta ``o que tinha nessa iteração mesmo'' e eu não lembrava.
Depois que começamos a usar o \calopsita{} ficou mais fácil, mas ainda hoje acho que tem bastante espaço pra melhorar, por exemplo saber o progresso das histórias, ter um ok, ``pode testar'' dos desenvolvedores.
Outra coisa que foi MUITO LEGAL, imprescindível, foi ter os builds ``estável'' e ``instável'' no ar pra gente poder brincar. Se tivesse que rodar na nossa máquina ou só quando encontrasse a equipe ia ser bem mais difícil manter o contato e comunicação que a gente manteve. \\

\textbf{\calopsita{}}: Dada sua visão inicial, acha que atendemos as expectativas? Acha que temos, hoje, um sistema que pode substituir o que vinha sendo usado? Que é mais usável?\\

\textbf{Hugo Corbucci}: Acho que a visão inicial mudou muito. De uma forma, sim, completamente satisfeito pelo que foi realizado. Ficou muito legal. De outro lado, ainda faltam algumas pequenas coisas para eu conseguir usar o calopsita para nossos projetos internos.

A usabilidade é inquestionável, o \calopsita{} é um projeto que mostra \textbf{claramente} uns 15 anos de avanço em cima do \textit{XPlanner} do ponto de vista da usabilidade.

Sobre os detalhes que ainda não permitem uso dele em produção, temos uma necessidade específica de controle de tempo gasto em cada tarefa que ainda não está desenvolvido no calopsita. Também falta um trabalho de ``marketing'' no site do projeto para facilitar o deploy dele em outros sistemas (algo como um guia de instalação e uma lista de requisitos necessários). Por fim, falta a natural coragem para investir o tempo nessa mudança e mudar de projetos para não perder a linha de trabalho já existente no \textit{XPlanner}.\\

\textbf{Guilherme Silveira}: A expectativa de substituir a lousa por um projetor me parece hoje em dia inviável -- seja pelo calopsita ou qualquer outra ferramenta -- ainda não consigo ver o computador substituindo uma caneta por completo. Mas a expectativa de poder manter um projeto no calopsita, fora alguns detalhes, é viável.\\

\textbf{Mariana Bravo}: Em termos de usabilidade, bate de 100 a zero o anterior, que era o XPlanner. Mas isso é pq o XPlanner é mesmo muito ruim. O calopsita está bom, eu diria muito bom, a gente se preocupou com isso desde o início. Mas ainda tem espaço pra melhorar. Ainda bem!

Quanto a funcionalidades, o calopsita ainda não substitui o xplanner, mas é por pouca coisa. Ele não precisa ter tudo que o xplanner tem, mas pra gente uma das coisas mais importantes que usamos no xplanner é a marcação de horários. Pelo menos nesse momento, é a única coisa que eu enxergo que está faltando para conseguir migrar.

Mas o sistema de plugins promete, esse é um dos plugins que quero tentar fazer! A vantagem de ter clientes-desenvolvedores ;-)