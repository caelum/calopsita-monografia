\section{Apêndice II - Entrevista com os clientes}

Para conseguirmos extrair a visão dos clientes, fizemos a seguinte entrevista:

\textbf{\calopsita{}}: O que vc esperava do projeto no inicio do ano?

\textbf{Hugo Corbucci}: Quando começamos a conversar, eu imaginei um mundo ideal no qual, ao final do ano, poderíamos escrever cartões de história, colocá-los em iterações e marcar neles o trabalho realizado. Com esses dados, poderíamos ver o progresso do trabalho como um todo olhando para a iteração. De uma certa forma, esperava conseguir ter as principais funcionalidades que usamos no \textit{Xplanner} mas em um sistema mais novo com uma interface bem melhor. Acho que isso descreve bem minha primeira ``visão'' do projeto.

\textbf{\calopsita{}}: Como cliente, o que você achou do processo de desenvolvimento do \calopsita{}? Do que sentiu falta? O que achou mais legal?

\textbf{Hugo Corbucci}: O processo de desenvolvimento foi bem legal. Gostei bastante do ritmo que tivemos até agosto. Depois disso, tanto por ausência nossa quanto de vocês, ficou um pouco mais parado. Senti falta de estarmos mais próximos nos períodos de programação mesmo. O ambiente de deploy contínuo foi muito bom mas tivemos alguns problemas críticos que tardaram a ser resolvidos. A ideia de montar as \textit{personas} para conseguirmos conversar melhor foi muito legal. 

\textbf{\calopsita{}}: Dada sua visão inicial, acha que atendemos as expectativas? Acha que temos, hoje, um sistema que pode substituir o \textit{Xplanner}? Que é mais usavel?

\textbf{Hugo Corbucci}: Acho que a visão inicial mudou muito. De uma forma, sim, completamente satisfeito pelo que foi realizado. Ficou muito legal. De outro lado, ainda faltam algumas pequenas coisas para eu conseguir usar o calopsita para nossos projetos internos.

A usabilidade é inquestionável, o \calopsita{} é um projeto que mostra CLARAMENTE uns 15 anos de avanço em cima do \textit{Xplanner} do ponto de vista da usabilidade.

Sobre os detalhes que ainda não permitem uso dele em produção, temos uma necessidade específica de controle de tempo gasto em cada tarefa que ainda não está desenvolvido no calopsita. Também falta um trabalho de ``marketing" no site do projeto para facilitar o deploy dele em outros sistemas (algo como um guia de instalação e uma lista de requisitos necessários). Por fim, falta a natural coragem para investir o tempo nessa mudança e mudar de projetos para não perder a linha de trabalho já existente no \textit{Xplanner}.