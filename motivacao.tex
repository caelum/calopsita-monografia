\section{Motivação e público alvo}

Nosso público alvo concentra-se em times distríbuidos ágeis, mas sem se restringir a alguma matodologia em específico. A idéia, é que a ferramenta tenha a capacidade de se adaptar às necessidades do time, servindo de facilitadora da comunicação, ao invés de um simples aparato burocrático. Uma inspiração são os times de projetos open source que em sua maioria são distribuídos, e quase sempre não tem uma ferramenta similar para acompanhamento de suas atividades, ou quando tem, essas ferramentas são deficientes (faremos um comparativo disso mais pra frente)

Embora o Calopsita seja mais uma ferramenta, a idéia é que seu uso seja um facilitador das iterações entre os indivíduos de um determinado projeto, não contrariando um dos valores do manifesto ágil:

"Individuals and interactions over processes and tool"


