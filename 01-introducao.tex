\section{Introdução}
Projetos ágeis são aqueles que usam as metodologias ou métodos que seguem e otimizam os preceitos de agilidade descritos no Manifesto Ágil~\cite{manifesto}. Esses projetos são iterativos e se preocupam mais em atender às expectativas dos usuários, ainda que essas mudem com o decorrer do tempo, do que com seguir um planejamento feito ainda no início do processo de construção de um sistema.

Mais do que isso, a agilidade em está em romper barreiras que atrapalham tanto desenvolvedores quanto clientes, isto é, eliminar toda buracracia desnecessária. Está em comunicação direta e visibilidade do projeto tanto para os desenvolvedores quanto para os clientes, que começam, muito mais cedo, a usar uma versão inicial do \software{}. E essa versão inicial é incrementada a cada iteração, um período curto de tempo. 

O \calopsita{} é um projeto que nasceu da necessidade de se trabalhar e gerenciar diferentes projetos ágeis, cada um com suas particularidades. Em especial, surgiu da necessidade de se trabalhar com equipes distribuídas em projetos ágeis. 

Essa necessidade ficou evidente em projetos de consultoria da empresa na qual os três idealizadores do \calopsita{} trabalham: o \textit{Product Owner}~\cite{po}~\cite{scrum} dos projetos é externo, mas ainda tem que escrever cartões de histórias, priorizá-los e acompanhar o desenvolvimento da iteração, ainda que à distância.

Não é intenção do \calopsita{} substituir métricas coladas em paredes e quadros brancos, mas sim prover uma ferramenta que permita o desenvolvimento ágil distribuído -- tanto comercial quanto \opensource{}.  

Esse objetivo convergia para os mestrados de dois amigos, Hugo Corbucci e Mariana Bravo, então decidiu-se convidá-los para serem clientes do projeto -- convite esse que foi aceito prontamente. Além deles, a equipe contou com o apoio do orientador, que acompanhava o andamento do projeto pela lista de discussões e conversas esporádicas, e de alguns amigos da Caelum, que foram de fundamental importância nas discussões sobre a arquitetura do sistema e que, mais tarde, também tornaram-se clientes e passaram a requisitar novas funcionalidades e reportar \textit{bugs}.

Já desde abril, o sistema é usado para gerenciar seu próprio desenvolvimento e, hoje, auxilia no desenvolvimento de vários outros projetos, tanto da Caelum como pessoais. 

Nesse trabalho, há três temas principais. Primeiramente, o processo de desenvolvimento, isto é, quais foram os métodos ágeis adotados para gerenciamento do sistema, a importante interação com clientes e a infraestrutura que viabilizou o crescimento e estabilidade do sistema. 

Em seguida, virá uma parte mais técnica que explica as inovações presentes no código do \calopsita{}, de onde elas vieram, o que as motivou e as impressões sobre esses avanços no estado da arte do mercado Java.

Finalmente, aborda-se o sistema produzido, comparando-o com outras ferramentas de proposta similar já existentes, tanto \opensource{} quanto comerciais, o feedback dos clientes e os passos futuros a serem implementados.
