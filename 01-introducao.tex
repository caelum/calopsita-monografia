\section{Introdução}
O Calopsita é um projeto que nasceu da necessidade de se trabalhar e gerenciar diferentes projetos ágeis, cada um com suas particularidades. Em especial, surgiu da necessidade de se trabalhar com equipes distribuidas em projetos ágeis. 

Essa necessidade ficou evidente em projetos de consultoria da empresa na qual os três idealizadores do Calopsita trabalham: o Product Owner dos projetos é externo, mas ainda tem que escrever cartões de estórias, priorizá-los e acompanhar o desenvolvimento da iteração, ainda que à distância.

Não é intenção do Calopsita substituir métricas coladas em paredes e quadros brancos, mas sim prover uma ferramenta que permita o desenvolvimento ágil distribuido -- tanto comercial quanto \opensource.  

Esse objetivo convergia para os mestrados de dois amigos, Hugo Corbucci e Mariana Bravo, então resolvemos convidá-los para serem nossos clientes de projeto -- convite esse que foi aceito prontamente. Além deles, contamos com o apoio do nosso orientador, que acompanhava o andamento do projeto pela lista de discussões e conversas esporádicas, e com alguns amigos da Caelum, que foram de fundamental importância nas discussões sobre a arquitetura do sistema e que, mais tarde, também tornaram-se clientes e passaram a pedir por novas funcionalidades e reportar bugs. 

Já desde abril, o sistema é usado para gerenciar seu próprio desenvolvimento e, hoje, auxilia no desenvolvimento de vários outros projetos, tanto da Caelum como pessoais.

Nesse trabalho, temos três temas principais relacionados ao Calopsita. Primeiramente, o processo de desenvolvimento, isto é, quais foram os métodos ágeis adotados para gerenciamento do sistema, a importante interação com clientes e a infraestrutura que viabilizou o crescimento e estabilidade do sistema. 

Em seguida, veremos uma parte mais técnica que explica as inovações presentes no código do Calopsita, de onde elas vieram, o que as motivou e nossas impressões sobre esses avanços no estado da arte do mercado Java.

Finalmente, falaremos do sistema produzido, comparando-o com outras ferramentas de proposta similar já existentes, \opensource e comerciais, do feedback dos clientes atuais e dos passos futuros a serem implementados.
