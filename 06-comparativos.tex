\section{Visão dos clientes e comparativos}

\subsection{Visão dos clientes}


\subsection{Comparativo com outras ferramentas}

Antes de iniciar o desenvolvimento do \calopsita{}, resolvemos analisar quais seriam as outras alternativas disponíveis no mercado. Fizemos uma análise bem extensa, tanto de produtos pagos, quanto de livres e \opensource{}. A idéia era descobrir quais os pontos fortes de cada ferramenta, e implementá-los no \calopsita{}. Embora as ferramentas pagas não sejam exatamente concorrentes, mostrar que nosso sistema tem as mesmas ou mais funcionalidades que as alternativas pagas, pode fazer com que atraiamos mais usuários e colaboradores.

\begin{itemize}
\item{Scrumy - https://scrumy.com/

É uma ferramenta paga e proprietária, com uma interface baseada em \textit{drag 'n' drop}. Não permite nenhum tipo de
customização, nem estimativas, marcação de horas, uso de \textit{templates} ou \textit{personas}. Quanto aos gráficos, o único disponibilizado
é o \textit{burndown}. Como diferencial, permite a visualização do
projeto em algum ponto do passado, atualizando o kanban para representar o estado do projeto naquele ponto, além de
permitir atualizações automáticas, ou seja, caso dois usuários estejam mexendo nele ao mesmo tempo, um enxerga as alterações
do outro sem precisar recarregar a página.}

\item{ScrumNinja - http://scrumninja.com

É uma ferramenta paga e proprietária, bem pouco customizável, sem a possibilidade de marcação de horas nem de
trabalhar com \textit{templates} ou \textit{personas}. Tem suporte a estimativas, uma interface baseada em estados (start, deliver, etc) e
gráficos de progresso. Um diferencial está no fato de que possui uma API que pode ser utilizada para a atualização dos cartões.}

\item{Scrum'd - http://scrumd.com/

É uma ferramenta paga e proprietária, não customizável, que permite a estimativa de histórias em pontos e de tarefas
em horas. Faz uso de \textit{burndown} e não trabalha com \textit{templates} nem \textit{personas}. Como diferencial, permite a importação e exportação de tarefas e histórias.}

\item{Pivotal Tracker - http://www.pivotaltracker.com/

É uma ferramenta proprietária, porém gratuita, que permite a estimativa de estórias em pontos, além da marcação da velocidade do time. Também não trabalha com \textit{templates} nem \textit{personas}, mas como diferencial permite a importação e exportação de tarefas e histórias, bem como permite a classificação de uma história em bug/feature/chore/release.}

\item{XPlanner - xplanner.org/

É uma ferramenta livre e \opensource{}, que inclusive vem sendo utilizada pela disciplina de Programação Extrema.
Ele tem bastantes funcionalidades, mas no entanto, sua interface de uso não é das mais agradáveis. O uso do XPlanner
por todos do time foi também um fator motivador na criação do \calopsita{}. Quanto às suas funcionalidades, o XPlanner
permite o uso de estimativas e marcação de horas, mas não é customizável nem trabalha com \textit{templates} ou \textit{personas}.}

\item{VersionOne - http://www.versionone.com

Ferramenta paga  e proprietária, customizável e com suporte a marcação de velocidade, geração de \textit{burndown}. Não trabalha
com \textit{templates} nem \textit{personas}, mas tem como diferencial a possibilidade de planejamento de releases. É a única ferramenta analisada com suporte a diferentes metodologias.}

\item{Pronto - http://www.bluesoft.com.br/pronto-demo

Ferramenta \opensource{}, com código hospedado no GitHub~\footnote{http://github.com/andrefaria/pronto-agile}. Permite a marcação de horas trabalhadas, bem como a geração de gráficos (embora durante nossos testes, a geração de \textit{burndowns} apresentou problemas). Não permite customizações nem o uso de \textit{templates} ou \textit{personas}. Como
diferencial permite que usuários tenham perfis diferentes (\textit{product owner}, \textit{scrum master}, desenvolvedor, testador, etc) e
permite informar se a tarefa for concluída usando pareamento, apontando a dupla envolvida.}

\item{PPTS - http://ses-ppts.sourceforge.net/

É uma ferramenta \opensource{} que possibilita a priorização de tarefas, bem como saber quais pessoas estiveram envolvidas
em quais tarefas. Permite a geração de gráficos diversos, bem como a estimativa de tarefas e a geração de relatório com
a velocidade do time. Como diferencial, permite a customização de menus.}

\item{Rally - http://www.rallydev.com/

Ferramenta proprietária, que permite a customização por widgets e usuários com diferentes perfis. Um diferencial está
no fato de que permite integração com a IDE Eclipse.}

\item{Mingle - http://studios.thoughtworks.com/mingle-agile-project-management

Ferramenta proprietária criada pela ThoughtWorks. Tem geração de gráficos, estimativas, priorização, marcação de horas.
Permite uma certa customização, mas não permite o uso de \textit{templates} nem de \textit{personas}. Como ponto positivo, está o fato de
que se adapta a diferentes metodologias ágeis, como o \calopsita{}.}

\end{itemize}

\begin{tabular}{|l|l|l|l|l|l|l|l|l|l|l|}
	\hline
	\multicolumn{11}{|c|}{Aplicações similares} \\
	\hline
	                & A & B & C & D & E & F & G & H & I & J \\
	Calopsita       & X & X & X & X & X & * & X & * & X & * \\
	Scrumy          & - & - & X & - & X & - & - & - & - & - \\
	ScrumNinja      & - & - & X & X & X & - & X & - & - & - \\
	Scrum'd         & - & - & X & X & X & - & X & - & - & - \\
	Pivotal Tracker & X & - & X & X & X & X & X & - & - & - \\
	XPlanner        & X & X & X & X & X & X & - & - & - & - \\
	VersionOne      & - & - & X & X & X & X & X & X & - & - \\
	Pronto          & X & X & X & - & X & X & - & - & - & - \\
	PPTS            & X & X & X & X & X & X & X & - & - & - \\
	Rally           & - & - & X & X & X & X & X & - & - & - \\
	Mingle          & - & - & X & X & X & X & X & X & - & - \\
	\hline
	\multicolumn{11}{|l|}{A - Gratuito} \\
	\multicolumn{11}{|l|}{B - Opensource} \\
	\multicolumn{11}{|l|}{C - Gráficos} \\
	\multicolumn{11}{|l|}{D - Estimativas} \\
	\multicolumn{11}{|l|}{E - Priorização} \\
	\multicolumn{11}{|l|}{F - Marcação de horas} \\
	\multicolumn{11}{|l|}{G - Customizável} \\
	\multicolumn{11}{|l|}{H - \textit{Templates} de metodologias} \\
	\multicolumn{11}{|l|}{I - \textit{Templates} de cartões} \\
	\multicolumn{11}{|l|}{J - \textit{Personas}} \\
	\multicolumn{11}{|l|}{* - no \textit{backlog} para ser feito} \\
	\hline
\end{tabular}
