\section{Visão dos clientes e comparativos}

clientes: mari, hugo, caelum

Metodologias

XP
Lean
Scrum
Crystal


Antes de iniciar o desenvolvimento do \calopsita, resolvemos analisar quais seriam as outras alternativas disponíveis no mercado. Fizemos uma análise bem extensa, tanto de produtos pagos, quanto de livres e \opensource. A idéia era descobrir quais os pontos fortes de cada ferramenta, e implementá-los no \calopsita. Embora as ferramentas pagas não sejam exatamente concorrentes, mostrar que nosso sistema tem as mesmas ou mais funcionalidades que as alternativas pagas, pode fazer com que atraiamos mais usuários e colaboradores.

\begin{itemize}
\item{Scrumy - https://scrumy.com/

É uma ferramenta paga e proprietária, com uma interface baseada em \textit{drag'n drop}. Não permite nenhum tipo de
customização, nem estimativas, marcação de horas, uso de templates ou personas. Quanto aos gráficos, o único disponibilizado
é o \textit{burndown}. Como diferencial, permite a visualização do
projeto em algum ponto do passado, atualizando o kanban para representar o estado do projeto naquele ponto, além de
permitir atualizações automáticas, ou seja, caso dois usuários estejam mexendo nele ao mesmo tempo, um enxerga as alterações
do outro sem precisar recarregar a página.}

\item{ScrumNinja - http://scrumninja.com

É uma ferramenta paga e proprietária, bem pouco customizável, sem a possibilidade de marcação de horas nem de
trabalhar com templates ou personas. Tem suporte a estimativas, uma interface baseada em estados (start, deliver, etc) e
gráficos de progresso. Um diferencial está no fato de que possui uma API que pode ser utilizada para a atualização dos
cartões.}

\item{Scrum'd - http://scrumd.com/

É uma ferramenta paga e proprietária, não customizável, que permite a estimativa de histórias em pontos e de tarefas
em horas. Faz uso de \textit{burndown} e não trabalha com templates nem personas. Como diferencial, permite a importação e
exportação de tarefas e estórias.}

\item{Pivotal Tracker - http://www.pivotaltracker.com/

É uma ferramenta proprietária, porém gratuita, que permite a estimativa de estórias em pontos, além da
marcação da velocidade do time. Também não trabalha com templates nem personas, mas como diferencial permite a importação e
exportação de tarefas e estórias, bem como permite a classificação de uma estória em bug/feature/chore/release.}

\item{XPlanner - xplanner.org/

É uma ferramenta free e open source, que inclusive vem sendo utilizada pela disciplina de Programação Extrema.
Ele tem bastantes funcionalidades, mas no entanto, sua interface de uso não é das mais agradáveis. O uso do XPlanner
por todos do time foi também um fator motivador na criação do \calopsita. Quanto às suas funcionalidades, o XPlanner
permite o uso de estimativas e marcação de horas, mas não é customizável nem trabalha com templates ou personas.}

\item VersionOne - http://www.versionone.com
**varias metodologias**
customizavel
burndown
velocidade
planejamento releases
nao trabalha com templates nem personas

\item Pronto - http://www.bluesoft.com.br/pronto-demo
usuarios com perfis diferentes (po, sm, dev, test, support)
burndown com problemas
priorizacao
lançamento de esforço
informação sobre se foi pareio
sem templates nem personas
nao customizavel
codigo fonte em portugues
livre / open source (github)

\item PPTS - http://ses-ppts.sourceforge.net/
free/open source
prioritizacao
burndown / graficos
velocidade / estimativas
controle de tarefas por pessoas
customização de menus
sem templates nem personas

\item AgileTrac - http://www.agile-trac.org
sistema de tickets
free/open source
sem graficos
sem customizações
apenas abas de iterações e backlog
medida de tamanha por pontos
acompanhamento de progresso por barra de percentagem x percento pronto

\item ScrumWorks

\item Rally
free / pro version - proprietario
integração com Eclipse =D
customização por widgets
usuarios com perfis diferentes (gerentes, dev, test, po)
graficos

\item Mingle

\end{itemize}

\begin{sidewaystable}
	\begin{tabular}{|l|l|l|l|l|l|l|l|}
		\hline
		\multicolumn{8}{|c|}{Aplicações similares} \\
		\hline
		 & Calopsita & Scrumy & ScrumNinja & Scrum'd & Pivotal Tracker & XPlanner & \\
		Gratuito & X & - & - & - & X & X & \\
		OpenSource & X & - & - & - & - & X & \\
		Graficos & X & X & X & X & X & X & \\
		Estimativas & X & - & X & X & X & X & \\
		Priorizacao & X & X & X & X & X & X & \\
		Marcacao Horas & F & - & - & - & X & X & \\
		Customizavel & X & - & X & - & X & - & \\
		Templates Metodologias & F & - & - & - & - & - & \\
		Templates Cartão & X & - & - & - & - & - & \\
		Personas & F & - & - & - & - & - & \\
		\hline
	\end{tabular}
\end{sidewaystable}


