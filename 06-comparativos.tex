\section{Visão dos clientes e comparativos}

\subsection{Visão dos clientes}

A fim de obter \textit{feedback} de nossos clientes quanto às atividades realizadas durante o desenvolvimento desse projeto,
um questionário~(\ref{entrevista}) sobre visão inicial, processo e resultados foi elaborado.

Quanto à visão inicial, a Mari e o Hugo imaginavam um sistema que substituísse o que eles vinham usando atualmente (\textit{XPlanner}). O foco para os dois era portanto, oferecer uma solução com melhor usabilidade. Já o Guilherme, imaginava poder substituir o uso de cartões físicos e do quadro branco pelo \calopsita{} (na Caelum, alguns projetos ocorrem com o time distribuído entre suas diversas unidades).

Já sobre o processo de desenvolvimento, a Mari e o Hugo disseram terem gostado bastante do processo de uma maneira geral. Gostaram de poder gerenciar o \calopsita{} usando o \calopsita{}, de terem criado \textit{personas} para a criação de histórias e do \textit{deploy} automatizado, que permitia que eles acompanhassem os trabalhos em tempo real. Por outro lado, algumas críticas foram feitas em relação à distância dos times nos momentos de programação e a dificuldade em saber o que estava pronto antes de utilizarmos o \calopsita{} para o gerenciamento do projeto.

Por fim, foram questionados sobre os resultados obtidos e se mostraram extremamente satisfeitos, mesmo reconhecendo que o sistema ainda tem muito o que melhorar. Disseram que o uso do \calopsita{} é bastante viável e que em termos de usabilidade já supera a ferramenta anterior, como enfatizado pelo Hugo na seguinte frase:

\begin{quote}
``A usabilidade é inquestionável, o \calopsita{} é um projeto que mostra CLARAMENTE uns 15 anos de avanço em cima do \textit{XPlanner} do ponto de vista da usabilidade.''
\end{quote}

\subsection{Comparativo com outras ferramentas}

Antes de iniciar o desenvolvimento do \calopsita{}, analisou-se outras alternativas de proposta semelhante disponíveis no mercado. Uma análise bem extensa foi feita, tanto de produtos pagos, quanto de gratuitos e \opensource{}. A intenção é descobrir quais os pontos fortes de cada ferramenta e implementá-los no \calopsita{}. 

Embora as ferramentas pagas não sejam exatamente concorrentes, dado o grande investimento que é feito nelas, mostrar que o sistema desenvolvido tem as mesmas ou mais funcionalidades que as alternativas pagas pode fazer com que mais usuários e colaboradores sejam atraídos.

\begin{itemize}
\item{\textbf{Scrumy - https://scrumy.com/}

É uma ferramenta paga e proprietária, com uma interface baseada em \textit{drag 'n' drop}. Não permite nenhum tipo de personalização, nem estimativas, marcação de horas, uso de \textit{templates} ou \textit{personas}. Quanto aos gráficos, o único disponibilizado é o \textit{burndown}. 

Como diferencial, permite a visualização do projeto em algum ponto do passado, atualizando o kanban para representar o estado do projeto naquele ponto, além de permitir atualizações automáticas, ou seja, caso dois usuários estejam mexendo nele ao mesmo tempo, um enxerga as alterações do outro sem precisar recarregar a página.}

\item{\textbf{ScrumNinja - http://scrumninja.com}

É uma ferramenta paga e proprietária, bem pouco personalizável, sem a possibilidade de marcação de horas nem de trabalhar com \textit{templates} ou \textit{personas}. Tem suporte a estimativas, uma interface baseada em estados (start, deliver, etc) e gráficos de progresso. 

Um diferencial está no fato de que possui uma API que pode ser utilizada para a atualização dos cartões.}

\item{\textbf{Scrum'd - http://scrumd.com/}

É uma ferramenta paga e proprietária, não personalizável, que permite a estimativa de histórias em pontos e de tarefas em horas. Faz uso de \textit{burndown} e não trabalha com \textit{templates} nem \textit{personas}. 

Como diferencial, permite a importação e exportação de tarefas e histórias.}

\item{\textbf{Pivotal Tracker - http://www.pivotaltracker.com/}

É uma ferramenta proprietária, porém gratuita, que permite a estimativa de histórias em pontos e a marcação da velocidade do time. Também não trabalha com \textit{templates} nem \textit{personas}. 

Como diferencial permite a importação e exportação de tarefas e histórias, assim como permite a classificação de uma história em \textit{bug/feature/chore/release}.}

\item{\textbf{XPlanner - xplanner.org/}

É uma ferramenta livre e \opensource{}, que vem sendo utilizada pela disciplina de Programação Extrema na USP. Ele tem muitas funcionalidades, no entanto sua interface de uso não é das mais agradáveis. O uso do XPlanner por todos do time foi também um fator motivador na criação do \calopsita{}. Esse sistema não é personalizável nem trabalha com \textit{templates} ou \textit{personas}.

Quanto aos seus pontos positivos, o XPlanner permite o uso de estimativas e marcação de horas.}

\item{\textbf{VersionOne - http://www.versionone.com}

Ferramenta paga  e proprietária, personalizável e com suporte a marcação de velocidade, geração de \textit{burndown}. Não trabalha com \textit{templates} nem \textit{personas}.

Tem como diferencial a possibilidade de planejamento de releases. É a única ferramenta analisada com suporte a diferentes metodologias.}

\item{\textbf{Pronto - http://www.bluesoft.com.br/pronto-demo}

Ferramenta \opensource{} e brasileira. Permite a marcação de horas trabalhadas, bem como a geração de gráficos (embora, durante os testes, a geração de \textit{burndowns} tenha apresentado problemas). Não permite customizações nem o uso de \textit{templates} ou \textit{personas}. 

Como diferencial permite que usuários tenham perfis diferentes (\textit{product owner}, \textit{scrum master}, desenvolvedor, testador, etc) e permite informar se a tarefa for concluída usando pareamento, apontando a dupla envolvida.}

\item{\textbf{PPTS - http://ses-ppts.sourceforge.net/}

É uma ferramenta \opensource{} que possibilita a priorização de tarefas, bem como permite saber quais pessoas estiveram envolvidas em quais tarefas. Permite a geração de gráficos diversos, bem como a estimativa de tarefas e a geração de relatório com a velocidade do time. 

Como diferencial, permite a customização de menus.}

\item{\textbf{Rally - http://www.rallydev.com/}

Ferramenta proprietária, que permite a customização por widgets e usuários com diferentes perfis. 

Um diferencial está no fato de que permite integração com a IDE Eclipse.}

\item{\textbf{Mingle - http://studios.thoughtworks.com/mingle-agile-project-management}

Ferramenta proprietária que conta com a geração de gráficos, estimativas, priorização, marcação de horas. Permite uma certa customização, mas não permite o uso de \textit{templates} nem de \textit{personas}. 

Como ponto positivo, está o fato de que se adapta a diferentes metodologias ágeis e possui integração com o \textit{Google Wave}.}

\end{itemize}

\subsection{Quadro comparativo}

No quadro comparativo a seguir pode-se visualizar de maneira mais simples quais ferramentas têm e quais não têm algumas características levantadas como importantes para a equipe do \calopsita{}. 

\begin{tabular}{|l|l|l|l|l|l|l|l|l|l|l|}
	\hline
	\multicolumn{11}{|c|}{Aplicações similares} \\
	\hline
	                & A & B & C & D & E & F & G & H & I & J \\
	Calopsita       & X & X & X & X & X & * & X & * & X & * \\
	Scrumy          & - & - & - & X & X & - & - & - & - & - \\
	ScrumNinja      & - & - & X & X & X & - & X & - & - & - \\
	Scrum'd         & - & - & X & X & X & - & X & - & - & - \\
	Pivotal Tracker & X & - & X & X & X & X & X & - & - & - \\
	XPlanner        & X & X & - & X & X & X & X & - & - & - \\
	VersionOne      & - & - & X & X & X & X & X & X & - & - \\
	Pronto          & X & X & - & X & X & X & - & - & - & - \\
	PPTS            & X & X & X & X & X & X & X & - & - & - \\
	Rally           & - & - & X & X & X & X & X & - & - & - \\
	Mingle          & - & - & X & X & X & X & X & X & - & - \\
	\hline
	\multicolumn{11}{l}{\textbf{Legenda:}} \\
	\multicolumn{4}{l}{A - Gratuito} & \multicolumn{7}{l}{F - Marcação de horas} \\
	\multicolumn{4}{l}{B - Opensource} & \multicolumn{7}{l}{G - Estimativas} \\
	\multicolumn{4}{l}{C - Personalizável} & \multicolumn{7}{l}{H - \textit{Templates} de metodologias} \\
	\multicolumn{4}{l}{D - Gráficos} & \multicolumn{7}{l}{I - \textit{Templates} de cartões} \\
	\multicolumn{4}{l}{E - Priorização} & \multicolumn{7}{l}{J - \textit{Personas}} \\
	\multicolumn{11}{l}{* - no \textit{backlog} para ser feito} \\
\end{tabular}
